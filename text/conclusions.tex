% main.tex

\section{Conclusions}

In this thesis we addressed exact results in supersymmetric gauge theories, more explicitly results valid at arbitrary values of the coupling constant in the planar limit.
The magic ingredients that helped us do all of this were the AdS/CFT dualities and their integrability.
Let us now summarize the structures we uncovered while exploring these theories and the results they helped us find.

For the major part of the thesis we considered $\N=4$ super Yang-Mills, which is a canonical example of a supersymmetric gauge theory with the ingredients listed above.
First of all it is dual to type IIB strings on $\adsfive$, which is an immensely powerful statement in its own right.
Furthermore both of these theories are integrable, which technically means that one can find as much conserved charges of motion as there are degrees of freedom.
We uncovered integrability on the gauge theory at weak coupling first when in section \ref{sec:integrability_weak} we mapped local gauge invariant operators to long ranged spin chains and the anomalous dimensions of the operators to the energies of the spin chain states.
Remarkably the abstract problem of finding the spectrum of local operators in a conformal field theory is equivalent to finding the spectrum of a very physical spin chain system.
It was at weak coupling where we also found the first exact result, the leading order small spin expansion coefficient of the generalized Konishi anomalous dimension, also known as the slope function. 
Of course it is a lucky discovery because the slope function is not sensitive to finite size effects, which are the major problem plaguing the spin chain picture.
They manifest as spin chain interactions that wrap around the chain.

At strong coupling $\N=4$ super Yang-Mills is more naturally described by type IIB strings on an $\adsfive$ background via the AdS/CFT correspondence. 
We uncovered integrability in this regime as well in section \ref{sec:integrability_strong}.
Here we found a picture of classical spectral curves, which are Riemann surfaces with sheets connected by branch cuts. 
String solutions were then described by quasi-momenta defined on these surfaces and the physical intuition was that each cut corresponded to an excitation of the string, where the sheets connected denoted the polarization, the size of the cut -- the amplitude and the discontinuity going through the branch cut -- the mode number of the excitation.
Amazingly the highly non-trivial motion of strings on a curved background somehow reduced to a description similar to that of a collection of harmonic oscillators.

The highlight of the whole integrability programme is of course the finite coupling regime where the two seemingly different descriptions meet, obviously they have to somehow emerge as limiting cases of some ultimate underlying structure.
In section \ref{sec:pmu_system} we argued that this hidden construction is the quantum algebraic curve. 
By now there are many ways of seeing its emergence from various limits, yet probably the most intuitive explanation is that the classical spectral curve is a type of WKB approximation of the quantum spectral curve, namely the classical quasi-momentum has to be replaced by its quantum version and the analyticity structure of the construction modified.
In a very rough sense the collection of classical harmonic oscillators gets quantized and the quantum quasi-momenta $\bP_a$ can be though of as the wave functions. 

The quantum spectral curve allowed us to rather easily derive the slope function and extend the calculation by finding the next coefficient in the small spin expansion, which we called the curvature function.
Contrary to the slope, the curvature function is sensitive to finite size effects and we indeed found evidence that our result successfully incorporates all of them.
We then used these exact results to find the first three coefficients in the strong coupling expansion of the Konishi anomalous dimension.
We matched them perfectly with available exact numeric data.
Another observable we discussed was the cusped Wilson line, whose expectation value contains the ubiquitous quantity often called the cusp anomalous dimension.
We showed how easily it can be found in the near BPS limit using the quantum spectral curve, even though technically the observable is non-local and the applicability of the construction is naively questionable.
Our exploration of the classical limit at strong coupling and the corresponding dual open string solution provided further evidence that this class of operators is not that different from local operators, as the result was basically a classical spectral curve.

Finally we touched up the ABJM theory which is another famous integrable supersymmetric gauge theory with a string dual.
The quantum spectral curve is also available in this setting and we briefly mentioned a very powerful exact result found with its help, the so-called interpolating function $h(\lambda)$, which basically encodes the relationship between the integrability coupling constant $h$ and the gauge theory coupling constant $\lambda$.
Amazingly this calculation even shed light on possible generalizations of the quantum spectral curve beyond the planar limit.
Roughly speaking the branch cuts of the quantum spectral curve here emerged as condensations of eigenvalues in the complex plane of an $N \times N$ sized matrix.
One can thus cautiously suspect that maybe going beyond the planar limit simply amounts to discretizing the branch cuts of the quantum spectral curve, a second quantization of sorts. 
Obviously at the moment these are just wild speculations.

In conclusion, AdS/CFT dualities and integrability are immensely powerful tools when dealing with gauge theories, so powerful that they enable one to find results exact in the coupling constant.
Obviously the ultimate goal of this research programme is to learn something that could be applied to real world theories such as QCD.
If we knew as much about QCD as we do about $\N=4$ super Yang-Mills by now we could analytically find the mass of the proton.
Unfortunately we are not there yet, but one can dream.




