% main.tex

\section{$\N=4$ super Yang-Mills}
\label{sec:cft}

\begin{chapquote}{German Proverb}
The devil is in the details.
\end{chapquote}

\noindent For the most part of this thesis we will be dealing with $\N=4$ super Yang-Mills theory. 
In this section we start off by defining it via its action and discussing its symmetries and observables. 
We also give an alternative formulation of the theory as a string theory, which is the core idea of the AdS/CFT correspondence. 
This formulation will later prove to be incredibly useful when discussing integrability and exact solutions.

\subsection{Action}

$\N=4$ super Yang-Mills theory is a quantum field theory much like the Standard Model of particle physics with a certain field content and interaction pattern.
It was first discovered by considering $\N=1$ super Yang-Mills theory in $9+1$ spacetime dimensions \cite{Brink:1977}, its action is given by
\begin{equation}
	S = \int d^{10} x \, \mathrm{Tr} \, \left( -\frac{1}{4}  F_{MN} F^{MN}  + \frac{1}{2} \bar{\Psi} \Gamma^M \mathcal{D}_M \Psi \right), \; \quad \; M = 1 \dots 10,
\end{equation}
where $\Psi$ is a Majorana-Weyl spinor in $9+1$ dimensions with $16$ real components and $\Gamma^M$ are the appropriate gamma matrices. 
The covariant derivative $\mathcal{D}_M$ is defined as
\begin{equation}
	\mathcal{D}_M = \partial_M - ig_{YM} \; [A_M, \, ],
\end{equation}
where $g_{YM}$ is the Yang-Mills coupling constant. 
The gauge group is in principle arbitrary, but we choose $SU(N)$ in anticipation of the AdS/CFT correspondence. 
By dimensionally reducing this theory on a flat torus $T^6$ one recovers the maximally supersymmetric $\N=4$ Yang-Mills gauge theory in $3+1$ spacetime dimensions.
The reduced action reads
\begin{equation}
\begin{split}
S = & \int d^4 x \, \tr \, \left( -\frac{1}{2} \mathcal{D}_\mu \Phi_I \mathcal{D}^\mu \Phi^I  + \frac{g_{YM}^2}{4} [\Phi_I, \Phi_J] [\Phi^I, \Phi^J] - \frac{1}{4}  F_{\mu\nu} F^{\mu\nu}  \right. \\
	 &\left. - \bar{\psi}^a \sigma^\mu \mathcal{D}_\mu \psi_a  + \frac{ig_{YM}}{2} \sigma^{ab}_I \psi_a [\Phi^I, \psi_b] + \frac{ig_{YM}}{2} \sigma_{ab}^I \bar{\psi}^a [\Phi_I, \bar{\psi}^b] \right).
\end{split}
\label{eq:n4_action}
\end{equation}
After dimensional reduction the gauge field $A_M$ decomposes to the four dimensional gauge field $A_\mu$ and to six real scalar fields $\Phi_I$ whereas the Majorana-Weyl spinor $\Psi_A$ breaks up into four copies of the left and right Weyl spinors in four dimensions 
\begin{equation}
	\Psi_A \, \, (A = 1, ..., 16) \, \, \, \rightarrow \, \, \,  \bar{\psi}^a_{\dot{\alpha}}, \; \psi_{a\alpha} \, \, (\alpha, \dot{\alpha} = 1, 2, \; \; a = 1, ..., 4).
\end{equation}
It also gives rise to the $SO(6) \simeq SU(4)$ symmetry called \emph{R-symmetry}, which originally was part of the ten dimensional Poincare group, but now acts as an internal symmetry of the supercharges.
It permutes the scalars, which live in the fundamental ${\bf 6}$ of $SO(6)$ and the spinors, which live in the fundamental of $SU(4)$, namely the lower index $a$ in $\psi_{a\alpha}$ transforms in ${\bf 4}$, while $\bar{\psi}_{\dot{\alpha}}^a$ transforms in ${\bf \bar{4}}$. 
From this it follows that we can combine the six real scalars $\Phi^I$ into three complex scalars $\Phi^{ab}$, often denoted as $X$, $Y$ and $Z$, which then transform under the second rank antisymmetric ${\bf 6}$ of $SU(4)$. 
The gauge field is a singlet under R-symmetry.

It is now a straightforward but rather tedious task to calculate the beta function for this theory. 
For any $SU(N)$ gauge theory at one loop level it is given by \cite{Gross:1973}
\begin{equation}
	\beta(g) = - \frac{g_{YM}^3}{16\pi^2} \left( \frac{11}{3} N - \frac{1}{6} \sum_s C_s - \frac{1}{3} \sum_f \tilde{C}_f \right)
\end{equation}
where the first sum is over the real scalars and the second one over the fermions. 
$C_s$ and $\tilde{C}_f$ are the quadratic Casimirs, which in our case are equal to $N$ since all fields are in the adjoint representation of the group. 
It is then easy to see that at least at one loop level the theory is conformally invariant. 
In fact the $\beta$ function was shown to be identically zero to all orders in perturbation theory \cite{Mandelstam:1983, Brink:1983}, hence $\N=4$ super Yang-Mills is fully conformally invariant even after quantization. 
After discussing the full symmetry algebra of the theory and its representations we will give an elegant argument why this is true.

\subsection{Observables}

The theory has 16 on-shell degrees of freedom which make up the gauge multiplet of $\N=4$ supersymmetry, namely $(\Phi_I, \psi_a, A_\mu)$. 
Gauge invariant operators are then formed by taking traces over the gauge group. 
An important class of operators are the \emph{local operators}, which are traces of fields all evaluated at the same spacetime point. 
They have the general form
\begin{eqnarray}
	\mathcal{O}_{i_1 \mu i_2 \alpha \dots i_n \dots j_1 \nu \beta \dots j_n}(x) & = \tr \left[ \Phi_{i_1}(x) \mathcal{D}_\mu \Phi_{i_2}(x) \psi_\alpha(x) \dots \Phi_{i_n}(x) \right] \times \dots \nonumber \\
	& \dots \times \tr \left[ \Phi_{j_1}(x) \mathcal{D}_\nu \psi_\beta(x) \dots \Phi_{j_n}(x) \right]. 
\end{eqnarray} 
In this thesis we will be exclusively focusing on the planar limit, which is the limit when the number of colors $N$ is sent to infinity. 
Diagrams involving multi-trace operators are non-planar, hence suppressed in the large $N$ limit and therefore we will only be considering single trace operators.
An example of a non-local operator is the Wilson loop, given by
\beq
	W_L= \tr \, \( {\mathcal P}\exp\!\oint_C \! dt\(i  A\cdot\dot{x}+\vec\Phi\cdot\vec n\,|\dot x|\) \),
\eeq
which depends on the path $x^\mu(t)$ in spacetime, hence it is known as a \emph{line operator}. 
It also depends on the coupling to the scalar fields, which is encoded in the six-dimensional unit vector $\vec{n}(t)$. 
The scalar field term can also be understood by recalling that the scalar fields are a result of dimensional reduction from $9+1$ dimensions, thus the coupling vector $\vec{n}(t)$ together with the curve $x^\mu(t)$ make up a path $x^M(t)$ in $9+1$ dimensional spacetime. 
In later sections of the text we will be considering cusped Wilson lines with other operators inserted at the cusp.

We will be mostly working in these two classes of operators, however in principle one could go on and define surface operators, etc.

\subsection{Symmetry}

Conformal symmetry, supersymmetry and R-symmetry are a part of a bigger group $PSU(2,2|4)$, which is also known as the \emph{$\N=4$ superconformal group}. 
It is the full symmetry group of $\N=4$ super Yang-Mills and is unbroken by quantum corrections. 
It is an example of a \emph{supergroup}, i.e. a graded group containing bosonic and fermionic generators. 
The theory of supergroups is highly developed (see \cite{Beisert:2010kp}) and much of the techniques from studying bosonic groups carry over to supergroups with some additional complications, i.e. Dynkin diagrams, root spaces, weights etc. 

$PSU(2,2|4)$ has the bosonic subgroup of $SU(2,2) \times SU(4)$, where $SU(2,2) \simeq SO(2,4)$ is the conformal group in four dimensions and $SU(4) \simeq SO(6)$ is the R-symmetry. 
The conformal group has the Poincar\'{e} group as a subgroup, which has a total of 10 generators including four translations $P_\mu$ and six Lorentz transformations $M_{\mu\nu}$, in addition there is the generator for dilatations $D$ and four special conformal generators $K_\mu$. 
Their commutation relations read
\begin{eqnarray}
 &[D, M_{\mu\nu}] = 0 \; \; \; [D, P_\mu] = -i P_\mu \; \; \; [D, K_\mu] = +i K_\mu,  \nonumber\\
 &[M_{\mu\nu}, P_\lambda] = -i(\eta_{\mu\nu} P_\nu - \eta_{\lambda\nu} P_\mu) \; \; \; [M_{\mu\nu}, K_\lambda] = -i(\eta_{\mu\lambda} K_\nu - \eta_{\lambda\nu} K_\mu),  \nonumber\\
 &[P_\mu, K_\nu] = 2i(M_{\mu\nu} - \eta_{\mu\nu} D).
 \label{eq:conformal_group}
\end{eqnarray}
$\N=4$ supersymmetry has 16 supercharges $Q_{a\alpha}$ and $\tilde{Q}^a_{\dot{\alpha}}$ where $\alpha, \dot{\alpha} = 1, 2$ are the Weyl spinor indices and $a = 1,...,4$ are the R-symmetry indices. 
These generators have the usual commutation and anti-commutation relations with the Poincar\'{e} generators given by
\begin{eqnarray}
	& \{Q_{\alpha a}, \tilde{Q}^b_{\dot{\alpha}}\} = \gamma^\mu_{\alpha\dot{\alpha}} \delta_a^b P_\mu \; \; \; \{Q_{\alpha a}, Q_{\alpha b}\} = \{ \tilde{Q}^a_{\dot{\alpha}}, \tilde{Q}^b_{\dot{\alpha}} \} = 0, \nonumber\\
	& [M^{\mu\nu}, Q_{\alpha a}] = i \gamma^{\mu\nu}_{\alpha\beta} \epsilon^{\beta\gamma} Q_{\gamma a} \; \; \; [M^{\mu\nu}, \tilde{Q}^a_{\dot{\alpha}}] = i \gamma^{\mu\nu}_{\dot{\alpha}\dot{\beta}} \epsilon^{\dot{\beta}\dot{\gamma}} \tilde{Q}_{\dot{\gamma}}^a, \nonumber \\
	& [P_\mu, Q_{\alpha a}] = [P_\mu, \tilde{Q}^b_{\dot{\alpha}}] = 0,
\end{eqnarray} 
where $\gamma_{\alpha\beta}^{\mu\nu} = \gamma^{[\mu}_{\alpha\dot{\alpha}} \gamma^{\nu]}_{\beta\dot{\beta}} \epsilon^{\dot{\alpha}\dot{\beta}}$. 
Commutators between supercharges and the conformal generators are also non trivial and introduce new supercharges,
\begin{eqnarray}
	& [D, Q_{\alpha a}] = -\frac{i}{2} Q_{\alpha a} \; \; \; [D, \tilde{Q}_{\dot{\alpha}}^a] = -\frac{i}{2} \tilde{Q}_{\dot{\alpha}}^a, \nonumber \\
	& [K^\mu,  Q_{\alpha a}] = \gamma^\mu_{\alpha\dot{\alpha}} \epsilon^{\dot{\alpha} \dot{\beta}} \tilde{S}_{\dot{\beta} a} \; \; \; [K^\mu, \tilde{Q}_{\dot{\alpha}}^a] = \gamma^\mu_{\alpha\dot{\alpha}} \epsilon^{\alpha\beta} S_\beta^a,
	\label{eq:dq_commutators}
\end{eqnarray}
where $\tilde{S}_{\dot{\alpha} a}$ and $S_\alpha^a$ are the \emph{special conformal supercharges}. 
They have opposite \text{R-symmetry} representations compared to the usual supercharges. 
The special supercharges bring the total of supercharges to 32. 
The commutation and anti-commutation relations for the special conformal supercharges are very much like the ones for normal supercharges,
\begin{eqnarray}
	& \{S_{\alpha}^a, \tilde{S}_{\dot{\alpha} b}\} = \gamma^\mu_{\alpha\dot{\alpha}} \delta_b^a K_\mu \; \; \; \{S_{\alpha}^a, S_{\alpha}^b\} = \{ \tilde{S}_{\dot{\alpha} a}, \tilde{S}_{\dot{\alpha} b} \} = 0, \nonumber\\
	& [M^{\mu\nu}, S_{\alpha}^a] = i \gamma^{\mu\nu}_{\alpha\beta} \epsilon^{\beta\gamma} S_{\gamma}^a \; \; \; [M^{\mu\nu}, \tilde{S}_{\dot{\alpha} a}] = i \gamma^{\mu\nu}_{\dot{\alpha}\dot{\beta}} \epsilon^{\dot{\beta}\dot{\gamma}} \tilde{S}_{\dot{\gamma} a}, \nonumber \\
	& [K_\mu, S_{\alpha}^a] = [K_\mu, \tilde{S}_{\dot{\alpha} a}] = 0.
\end{eqnarray} 
Finally the anti-commutation relations between the special conformal and usual supercharges close the algebra,
\begin{eqnarray}
	\{ Q_{\alpha a}, S_\beta^b \} & = & - i \epsilon_{\alpha\beta} {{\sigma^{IJ}}_a}^b R_{IJ} + \gamma_{\alpha\beta}^{\mu\nu} {\delta_a}^b M_{\mu\nu} - \frac{1}{2} \epsilon_{\alpha\beta} {\delta_a}^b D \nonumber \\
	\{ \tilde{Q}_{\dot{\alpha}}^a, \tilde{S}_{\dot{\beta} b} \} & = & + i \epsilon_{\dot{\alpha}\dot{\beta}} {{\sigma^{IJ}}^a}_b R_{IJ} + \gamma_{\dot{\alpha}\dot{\beta}}^{\mu\nu} {\delta^a}_b M_{\mu\nu} - \frac{1}{2} \epsilon_{\dot{\alpha}\dot{\beta}} {\delta^a}_b D \nonumber \\
	\{ Q_{\alpha a}, \tilde{S}_{\dot{\beta} b} \} & = & \{ \tilde{Q}_{\dot{\alpha}}^a, S_\beta^b \} = 0
	\label{eq:qs_anticommutators}
\end{eqnarray}
where $R_{IJ}$ are the generators of R-symmetry with $I,J = 1, ..., 6$. 
All supercharges transform under the two spinor representations of the R-symmetry group and all other generators commute with it. 
All of the generators can be organized as follows
\beq
\(
	\begin{array}{c|c}
	K^\mu, P^\mu, M^{\mu\nu}, D & Q_{a\alpha}, \bar{S}_{a\dot{\alpha}} \\ \hline 
	S_{\alpha}^a, \bar{Q}_{\dot{\alpha}}^a  & R_{IJ}
	\end{array} 
\)
\eeq
where the generators in the diagonal blocks are bosonic and the ones in the anti-diagonal blocks are fermionic.
They have a definite dimensions, which are not modified by the radiative corrections
\beq
	[D]=[L]=[\bar L]=[R]=0\;, \quad [P]=1\;, \ [K]=-1\;, \quad [Q]=1/2\;,\  [S]=-1/2\;.
\eeq
In contrast, the classical dimensions of fields
\beq
	[\Phi^I] = [A_\mu] = 1\;, \quad [\psi_a] = \frac{3}{2},
\eeq
do receive radiative corrections and acquire \emph{anomalous dimensions}, which together with the bare dimension make up the conformal dimension
\beq
	\Delta = \Delta_0 + \gamma(g_{YM}).
\eeq
The name is justified by the fact that in conformal field theories all two point functions are determined by the scaling dimensions of the fields. 
More than that, together with the knowledge of all three point functions they are enough to determine any $n$-point function. 
This is why finding conformal dimensions of all operators, i.e. the spectrum of the theory is a very important step in solving it.

\subsubsection{Superconformal multiplets}

Fields of the theory can be organized in unitary representations of the superconformal symmetry group, which are labeled by quantum numbers of the bosonic subgroup
\beqa
	&SO(1,3) \times SO(1,1) \times SU(4) \nonumber \\
	&\quad \; (s_+, s_-) \quad \quad \Delta \quad \quad  [r_1, r_2, r_3]
\eeqa 
where $(s_+, s_-)$ are the usual positive half-integer spin labels of the Lorentz group, $\Delta$ is the positive conformal dimension that can depend on the coupling and $[r_1, r_2, r_3]$ are Dynkin labels of the $R$-symmetry.
All unitary representations of the superconformal group have been classified into four families \cite{Dobrev:1985ab,Dobrev:1985cd}, here we give a short description of the classification.
  
Looking at the commutation relations of the conformal subgroup (\ref{eq:conformal_group}), we see that the operators $P_\mu$ and $K_\mu$ act as raising and lowering operators for the dilatation operator $D$ -- this gives a hint as to how we could construct representations of the group. 
The dilatation operator $D$ is the generator of scalings, i.e. upon a rescaling $x \rightarrow \lambda x$ a local operator in a field theory scales as 
\begin{equation}
	\mathcal{O}(x) \rightarrow \lambda^{-\Delta} \mathcal{O}(\lambda x)
\end{equation}
where $\Delta$ is the conformal dimension of the operator $\mathcal{O}(x)$. 
Restricting to the point $x = 0$, which is a fixed point of scalings, we see that the conformal dimension is the eigenvalue of the dilatation operator,
\begin{equation}
	[D,\mathcal{O}(0)] = -i \Delta \mathcal{O}(0).
\end{equation}
It is now clear that acting on a field with $K_\mu$ should lower the dimension by one and acting with $P_\mu$ -- raise it by one. 
We can show this explicitly using the Jacobi identity as
\beq
	[D, [K_\mu, \mathcal{O}(0)]] = [[D, K_\mu], \mathcal{O}(0)] + [K_\mu, [D, \mathcal{O}(0)]] = -i (\Delta - 1) \; [K_\mu, \mathcal{O}(0)].
\eeq
Since operators in a unitary quantum field theory should have positive dimensions (aside from the identity operator), we should not be able to keep lowering the dimension indefinitely, i.e. there should always be an operator that satisfies
\begin{equation}
	[K_\mu, \tilde{\mathcal{O}}(0)] = 0.
\end{equation} 
We call such operators \emph{conformal primary operators}. 
Acting on these with $P_\mu$ keeps producing operators with a dimension one higher -- we call these the \emph{descendants} of $\tilde{\mathcal{O}}(0)$. 
We can also act with the supercharges and looking at the commutators in (\ref{eq:dq_commutators}) we see that they raise the dimension by $1/2$, while the special conformal supercharges lower it by $1/2$. 
Operators annihilated by special conformal supercharges are called \emph{superconformal primaries}, which is a stronger condition that being a conformal primary.

(Super-)conformal primaries and their descendants make up multiplets that constitute the three families of discrete representations in the classification.
They are further distinguished by the number of supercharges the primary commutes with.
One example is a class of operators that satisfy the condition 
\beq
	\label{eq:halfBPS}
	\Delta = r_1 + r_2 + r_3,
\eeq 
a canonical representative would be a single-trace symmetrized scalar field operator such as 
\beq
	\mathcal{O}^{i j \dots k}(x) = \tr \( \Phi(x)^{(i} \Phi(x)^j \dots \Phi(x)^{k)} \).
\eeq
These operators commute with half of the supercharges, thus they are referred to as \text{half-BPS}. 
A key fact is that operators in the same representation must have the same anomalous dimension, because the generators of the group can only change it by half integer steps and there's only a finite number of generators. 
What is more, operators in the discrete BPS representations are protected from quantum corrections, because at any coupling the total dimension is always algebraically related to the Dynkin labels of the R-symmetry, e.g. as in \eq{eq:halfBPS}. 
Since charges of compact groups are quantized it must meant that the dimension can't continuously depend on the coupling and hence the anomalous dimension must vanish. 
This is however not true for the fourth continuous non-BPS family of representations, hence operators from these multiplets do acquire anomalous dimensions.

Let us conclude the section with an elegant argument for why the beta function of $\N=4$ super Yang-Mills is zero.
One can use the algebra and shown that the operators $\tr \, F_+ F_+$ and $\tr \, F_- F_- $, where $F_\pm$ are the (anti-)self-dual field strengths, belong to the same multiplet as a superconformal primary \cite{Minahan:2010js}, meaning that the $\tr \, F_{\mu\nu} F^{\mu\nu}$ term in the Lagrangian is protected from quantum corrections, hence so is the coupling constant $g_{YM}$. 
This argument is valid to all orders in perturbation theory, which means that $\N=4$ super Yang-Mills is conformally invariant to all orders in perturbation theory.

\subsection{String description at strong coupling}
\label{sec:n4_strong}

As already briefly explained in the introduction, the AdS/CFT conjecture states that $\N=4$ super Yang-Mills is exactly dual to type IIB string theory on $\adsfive$, \cite{Maldacena:1997re,Gubser:1998bc,Witten:1998qj}. 
To be more precise, the gauge group of the Yang-Mills theory is taken to be $SU(N)$ and the coupling constant $g_{YM}$. 
The string theory is defined on $\adsfive$ where both $AdS_5$ and $S^5$ have radius $R$. 
The self-dual five-form field $F_5^+$ has integer flux through the sphere
\beq
	\int_{S^5} F_5^+ = N,
\eeq
and $N$ is identified with the number of colors in the gauge theory. 
The string theory is further parametrized by the string coupling $g_s$ and the string string length squared $\alpha'$. The following relations are conjectured to hold
\beq
	 4\pi g_s = g_{YM}^2 \equiv \frac{\lambda}{N}, \;\;\;\;\;\;\; \frac{R^4}{\alpha'^2} = \lambda,
\eeq
where $\lambda$ is the t'Hooft coupling. We will be working in the planar limit $N \rightarrow \infty$ with $\lambda$ fixed. 
It is easy to see that in this limit $g_s \rightarrow 0$ and we are left with freely propagating strings.
Furthermore, the regime of strongly coupled gauge theory when $\lambda \rightarrow \infty$ corresponds to the regime of string theory where the supegravity approximation is valid, namely $\alpha' \ll R^2$.
The takeaway here is that one can formulate strongly coupled planar $\N=4$ super Yang-Mills as a classical theory of free strings on $\adsfive$.


\subsubsection{Sigma model formulation}

A very useful formulation of string theory on $AdS_5 \times S^5$ is the coset space sigma model \cite{Metsaev:1998it} with the target superspace of
\begin{equation}
	\frac{PSU(2,2|4)}{SO(1,4) \times SO(5)}.
\end{equation}
\vspace{2pt}
The bosonic part of the supercoset where the string moves is given by
\beq
	\frac{SO(2,4) \times SO(6)}{SO(1,4) \times SO(5)} = \adsfive,
\eeq
which is constructed as the coset between the isometry and isotropy groups of $\adsfive$. 
The action is then written in terms of the algebra of $PSU(2,2|4)$.

The superalgebra $\alg{psu}{2,2|4}$ has no realization in terms of matrices, instead it is the quotient of $\alg{su}{2,2|4}$ by matrices proportional to the identity. 
On the other hand $\alg{su}{2,2|4}$ is a matrix superalgebra spanned by $8\times8$ supertraceless matrices
\beq
M = \( {\begin{array}{c|c}
 A & B  \\
 \hline
 C & D  \\
 \end{array} } \),
\eeq
where the supertrace is defined as 
\beq
\str M = \tr A - \tr D.
\eeq
$A$ and $D$ are elements of $\alg{su}{2,2}$ and $\alg{su}{4}$ respectively, whereas the fermionic components are related by
\beq
	C = \( {\begin{array}{c|c}
 +\mathbbm{1}_{2\times2} & 0  \\
 \hline
 0 & -\mathbbm{1}_{2\times2}  \\
 \end{array} } \) B^\dagger.
\eeq
An important feature of this algebra is the following automorphism
\beq
\Omega \circ M = \( {\begin{array}{c|c}
 E A^T E & -E C^T E  \\
 \hline
 E B^T E & E D^T E \\
 \end{array} } \),
 \; \quad  \;
E = \( {\begin{array}{cccc}
 0 & -1 & 0 & 0  \\
 1 & 0 & 0 & 0 \\
 0 & 0 & 0 & -1  \\
 0 & 0 & 1 & 0 \\
 \end{array} } \),
\eeq
which endows the algebra with a $\mathbb{Z}_4$ grading, since one can easily check that $\Omega^4 = 1$.
This in turn means that any element of the algebra can be decomposed under this grading as
\beq
	M = \sum_{i=0}^3 M^{(i)},
\eeq
where
\beqa
	\label{eq:M_components}
	M^{(0,2)} &= \frac{1}{2} \( {\begin{array}{c|c}
 A \;\, \pm E A^T E & 0  \\
 \hline
 0 & D \pm E D^T E  \nonumber \\
 \end{array} } \) \\
 M^{(1,3)} &= \frac{1}{2} \( {\begin{array}{c|c}
 0 & B \pm i E C^T E  \\
 \hline
 C \mp i E B^T E & 0  \\
 \end{array} } \)
\eeqa 
and the morphism then acts on the elements of the decomposition as
\beq
	\Omega \circ M^{(n)} = i^n M^{(n)}.
\eeq
The Metsaev-Tseytlin action for the Green-Schwarz superstring is then given by
\begin{equation}
	\label{eq:mt_action}
	S = \frac{\sqrt{\lambda}}{4 \pi} \int \str \left( J^{(2)} \wedge * J^{(2)} - J^{(1)} \wedge J^{(3)} + \Lambda \wedge J^{(2)} \right),
\end{equation}
which is written down in terms of the graded elements of the algebra current
\begin{equation}
	J = -g^{-1} \mathrm{d} g %\in \mathfrak{psu(2,2|4)},
	\label{eq:j_current}
\end{equation}
where $g(\sigma, \tau) \in PSU(2,2|4)$ is a map from the string worldsheet to the supergroup $PSU(2,2|4)$. 
The last term contains a Lagrange multiplier $\Lambda$, which ensures that $J^{(2)}$ is supertraceless, whereas all other components are manifestly traceless as seen from \eq{eq:M_components}. 
Since the target space is the coset of $PSU(2,2|4)$ by $SO(1,4) \times SO(5)$, the map $g$ has an extra gauge symmetry
\begin{equation}
	g \rightarrow gH, \,\,\,\,\, H \in SO(1,4) \times SO(5)
\end{equation}
under which the components of the supercurrent transform as
\beqa
	& J^{(0)} \rightarrow H^{-1} J^{(0)} H - H^{-1} \mathrm{d} H \\
	& J^{(i)} \rightarrow H^{-1} J^{(i)} H, \quad i=1,2,3
\eeqa
The equations of motion read
\beq
	d * k = 0,
\eeq
where $k = gKg^{-1}$ and
\beq
	K = J^{(2)} + \frac{1}{2} * J^{(1)} - \frac{1}{2} * J^{(3)} - \frac{1}{2} * \Lambda.
\eeq
They are equivalent to the conservation of the Noether current associated to the global left $PSU(2,2|4)$ multiplication symmetry.

Finally let us briefly remark on how the action reduces to the usual sigma model action if one restricts to bosonic fields. 
A purely bosonic representative of $PSU(2,2|4)$ has the form
\beq
	g = \( {\begin{array}{c|c}
 A & 0  \\
 \hline
 0 & D  \\
 \end{array} } \),
\eeq
where $A \in SO(6) \simeq SU(4)$ and $D \in SO(2,4) \simeq SU(2,2)$. Then we see that $A E A^T$ is a good parametrization of 
\beq
	\frac{SO(6)}{SO(5)} \simeq \frac{SU(4)}{SP(4)} = S^5,
\eeq
since it is invariant under $A \rightarrow A H$ with $H \in SP(4)$. 
Similarly $D E D^T$ parametrizes $AdS_5$. 
If we now define the coordinates $u^i$ and $v^i$ in the following way
\beq
	u^i \Gamma^S_i = A E A^T, \quad \quad v^i \Gamma^A_i = D E D^T,
\eeq
with $\Gamma^S$ and $\Gamma^A$ being the gamma matrices of $SO(6)$ and $SO(2,4)$ respectively, then by construction they will satisfy the following constraints
\beqa
	1 &= u \cdot u \equiv +u_1^2 + u_2^2 + u_3^2 + u_4^2 + u_5^2 + u_6^2 \nonumber \\
	1 &= v \cdot v \equiv -v_1^2 - v_2^2 - v_3^2 - v_4^2 + v_5^2 + v_6^2,
\eeqa
and the action \eq{eq:mt_action} will read
\beq
	S_b = \frac{\sqrt{\lambda}}{4\pi} \int_0^{2\pi} d\sigma \int d\tau \; \sqrt{h} \( h^{\mu\nu} \pd_\mu u \cdot \pd_\nu u + \lambda_u \( u \cdot u - 1 \) - \( u \rightarrow v \) \),
\eeq
which is of course just the usual non-linear sigma model for a string moving in $\adsfive$.
