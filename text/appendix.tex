% main.tex

\section{Summary of notation and definitions}
\label{sec:notations}

In this appendix we summarize some notation used ubiquitously throughout the thesis.

\subsection{Laurent expansions in $x$}

We often represent functions of the spectral parameter $u$ as a series in $x$
\beq
f(u)=\sum\limits_{n=-\infty}^{\infty}f_n x^n
\eeq
with
\beq
	u=g(x+1/x).
\eeq
We denote by $[f]_+$ and $[f]_-$ part of the series with positive and negative powers of $x$:
\beqa
&&[f]_+=\sum\limits_{n=1}^{\infty}f_n x^n, \\
&&[f]_-=\sum\limits_{n=1}^{\infty}f_{-n} x^{-n}.
\eeqa
As a function of $u$, $x(u)$ has a cut from $-2g$ to $2g$. For any function $f(u)$ with such a cut we denote another branch of $f(u)$ obtained by analytic continuation (from $\Im \;u>0$) around the branch point $u=2g$ by $\tilde f(u)$. In particular, $\tilde x=1/x$.

\subsection{Functions $\sinh_\pm$ and $\cosh_\pm$}

We define $I_k=I_k(4 \pi g)$, where $I_k(u)$ is the modified Bessel function of the first kind.
Then
\beqa
&& \sinh_+=[\sinh(2\pi u)]_+=\sum\limits_{k=1}^\infty I_{2k-1}x^{2k-1}, \\
&& \sinh_-=[\sinh(2\pi  u)]_-=\sum\limits_{k=1}^\infty I_{2k-1}x^{-2k+1},\\
&& \cosh_+=[\cosh(2\pi u)]_+=\sum\limits_{k=1}^\infty I_{2k}x^{2k}, \\
&& \cosh_-=[\cosh(2\pi u)]_-=\sum\limits_{k=1}^\infty I_{2k}x^{-2k}.
\eeqa
 In some cases we denote for brevity
\beq
	\sh_-^x=\sinh_-(x),\ \ \ \ch_-^x=\cosh_-(x).
\eeq

\subsection{Integral kernels}

In order to solve for $\bP_a^{(1)}$ in section \ref{sec:CalculationofPa} we introduce integral operators $H$ and $K$ with kernels
\beqa
H(u,v)&=&-\frac{1}{4\pi i}\frac{\sqrt{u-2g}\sqrt{u+2g}}{\sqrt{v-2g}\sqrt{v+2g}}\frac{1}{u-v}dv, \\
K(u,v)&=&+\frac{1}{4\pi i}\frac{1}{u-v}dv,
\eeqa
which satisfy
\beq
\tilde f+f=h\;\;,\;\;f=H\cdot h\;\;\;\;{\rm and}\;\;\;\;
\tilde f-f=h\;\;,\;\;f=K\cdot h.
\label{lab123}
\eeq
Since the purpose of $H$ and $K$ is to solve equations of the type \eq{lab123}, $H$ usually acts on functions $h$ such that $\tilde h=h$, whereas $K$ acts on $h$ such that $\tilde h=-h$. 
On the corresponding classes of functions $H$ and $K$ can be represented by kernels which are equal up to a sign
\beqa
H(u,v)&=&-\left.\frac{1}{2\pi i}\frac{1}{x_u-x_v}dx_v\right|_{\tilde h=h},\;\;\\
K(u,v)&=&\left.\frac{1}{2\pi i}\frac{1}{x_u-x_v}dx_v\right|_{\tilde h=-h}.
\eeqa
In order to be able to deal with series in half-integer powers of $x$ in section \ref{sec:SolvingPmuL3} we introduce modified kernels:
\beqa
&&H^*\cdot f\equiv\frac{x+1}{\sqrt{x}}H\cdot\frac{\sqrt{x}}{x+1} f, \\
&&K^*\cdot f\equiv\frac{x+1}{\sqrt{x}}K\cdot\frac{\sqrt{x}}{x+1} f.
\eeqa
Finally, to write the solution to equations of the type \eqref{eq:mudiscNLO}, we introduce the operator $\Gamma'$ and its more symmetric version $\Gamma$
\beq
\(\Gamma'\cdot h\)(u)\equiv \oint_{-2g}^{2g}\frac{dv}{{4\pi i}}\d_u \log \frac{\Gamma[i (u-v)+1]}{\Gamma[-i (u-v)]}h(v),
\eeq
\beq
\(\Gamma\cdot h\)(u)\equiv \oint_{-2g}^{2g}\frac{dv}{{4\pi i}}\d_u \log \frac{\Gamma[i (u-v)+1]}{\Gamma[-i (u-v)+1]}h(v).
\eeq

\subsection{Periodized Chebyshev polynomials}
\label{sec:appPeriodized}

Periodized Chebyshev polynomials appearing in $\mu_{ab}^{(1)}$ are defined as
\beqa
&&p_a'(u)=\Sigma\cdot\left[x^a+1/x^a\right]=2 \Sigma\cdot\left[T_a\(\frac{u}{2g}\)\right],\\
&&p_a(u)=p_a'(u)+\frac{1}{2}\(x^a(u)+x^{-a}(u)\),
\eeqa
where $T_a(u)$ are Chebyshev polynomials of the first kind. Here is the explicit form for the first five of them:	
\beqa
&&p_0'=-i(u-i/2),\\
&&p_1'=-i\frac{u(u-i)}{4g},\\
&&p_2'=-i\frac{(u-i/2)(-6g^2+u^2-iu)}{6g^2},\\
&&p_3'=-i\frac{ u (u-i) \left(-6 g^2+u (u-i)\right)}{8 g^3},\\
&&p_4'=-i\frac{ \left(u-\frac{i}{2}\right) \left(30 g^4-20 g^2 u^2+20 i g^2 u+3 u^4-6 i u^3-2 u^2-i u\right)}{30 g^4}.
\eeqa

\section{Slope function: details}

Here we fill in some of the details and provide generalizations for the $\pmu$-system solution of the slope function discussed in section \ref{sec:slope_pmu}.

\subsection{Solution for odd $J$}
\label{sec:oddL}

Here we give details on solving the $\bP\mu$-system for odd $J$ at leading order in the spin. 
First, the parity of the $\mu_{ab}$ functions is different from the even $J$ case, which can be seen from the asymptotics \eq{eq:muasymptotics}. 
Following arguments similar to the discussion for even $J$ in section \ref{sec:slope_pmu_solve}, we obtain
\beq
	\mu_{12}=1,\ \mu_{13}=0,\ \mu_{14}=0,\  \mu_{24}=\cosh(2\pi u),\ \mu_{34}=1.
\eeq
Plugging these $\mu_{ab}$ into \eqref{eq:pmuexpanded} we get a system of equations for $\bP_a$
\beqa
&&\tilde \bP_1= -\bP_3,  \\
&&\tilde \bP_2= -\bP_4 -\bP_1 \cosh(2\pi u), \\
&&\tilde \bP_3= -\bP_1,\\
&&\tilde \bP_4= -\bP_2+\bP_3 \cosh(2\pi u).
\eeqa
This system can be solved in a similar way to the even $J$ case. 
The only important difference is that due to asymptotics \eq{eq:asymptotics} the $\bP_a$ acquire an extra branch point at $u=\infty$.
Let us first rewrite the equations for $\bP_1,\bP_3$ as
\beqa
\tilde\bP_1+\tilde\bP_3&=&-\(\bP_1+\bP_3\)\\
\tilde\bP_1-\tilde\bP_3&=&\bP_1-\bP_3.
\eeqa
This, together with the asymptotics \eqref{eq:asymptotics} implies $\bP_1=\epsilon  x^{-J/2},\;\bP_3=-\epsilon  x^{J/2}$ where $\epsilon$ is a constant. 
Let us note that these $\bP_1, \bP_3$ contain half-integer powers of $x$, and the analytic continuation around the branch points at $\pm 2g$ replaces $\sqrt{x}\to1/\sqrt{x}$. 
Now, taking the sum and difference of the equations for $\bP_2,\;\bP_4$ we get
\beqa
&&\tilde\bP_2+\tilde\bP_4+\bP_2+\bP_4=-a_1\(x^{J/2}+x^{-J/2}\)\cosh{2\pi u},\\
&&\tilde\bP_2-\tilde\bP_4-\(\bP_2-\bP_4\)=a_1\(x^{J/2}-x^{-J/2}\)\cosh{2\pi u}.
\eeqa
We can split the expansion
\beq
	\cosh{2\pi u}=\sum\limits_{k=-\infty}^{\infty}I_{2k}x^{2k}
\eeq	
into the positive and negative parts according to
\beq
\cosh{2\pi u}=\cosh_-+\cosh_++I_0,
\eeq
where
\beq
\cosh_+=\sum\limits_{k=1}^{\infty}I_{2k}x^{2k},\;\ \ \ \ \cosh_-=\sum\limits_{k=1}^{\infty}I_{2k}x^{-2k}.
\eeq
Then we can write
\beqa
&&\bP_2+\bP_4=-a_1(x^{J/2}+x^{-J/2})\cosh_--a_1 I_0 x^{-J/2}+Q, \\
&&\bP_2-\bP_4=-a_1(x^{J/2}-x^{-J/2})\cosh_-+a_1 I_0 x^{-J/2}+P,
\eeqa
where $Q$ and $P$ are some polynomials in $\sqrt{x},1/\sqrt{x}$ satisfying
\beq\label{QP}
	\tilde Q=-Q,\; \tilde P=P.
\eeq
We get
\beqa
\label{eq:P2tmp}
&&\bP_2=-a_1 x^{J/2}\cosh_- +\frac{Q+P}{2},\\
\label{eq:P4tmp}
&&\bP_4=a_1 x^{-J/2}\cosh_- -a_1 I_0 x^{-J/2}+\frac{Q-P}{2}.
\eeqa
Now imposing the correct asymptotics of $\bP_2$ we find
\beq
\frac{P+Q}{2}=a_1 x^{J/2}\sum\limits_{k=1}^{\frac{J-1}{2}}I_{2k}x^{-2k}.
\eeq
Due to \eq{QP} this relation fixes $Q$ and $P$ completely, and we obtain the solution given in section \ref{sec:slope_pmu_solve},
\beqa
\label{eq:musolLOoddL}
&&	\mu_{12}=1,\ \mu_{13}=0,\ \mu_{14}=0, \ \mu_{24}=\cosh(2\pi u),\ \mu_{34}=1, \\
&&   \bP_1=a_1 x^{-J/2}, \\
&&   \bP_2=-a_1 x^{J/2}\sum\limits_{k=-\infty}^{-\frac{J+1}{2}}I_{2k}x^{2k},\\
&&   \bP_3=-a_1 x^{J/2}, \\
\label{eq:P4solLOoddL}
&&    \bP_4=a_1 x^{-J/2}\cosh_--a_1 x^{-J/2}\sum\limits_{k=1}^{\frac{J-1}{2}}I_{2k}x^{2k}-a_1 I_0 x^{-J/2}.
\eeqa
Notice that the branch point at infinity is absent from the product of any two $\bP$'s, as it should be \cite{Gromov:2013pga,Gromov:2014caa}. 
One can check that this solution gives again the correct result \eqref{eq:resultLO} for the slope function.

\subsection{Generic filling fractions and mode numbers}
\label{sec:Sanyn}

Let us extend the discussion of section \ref{sec:slope_pmu} by considering the state corresponding to a solution of the asymptotic Bethe equations with arbitrary mode numbers and filling fractions.
For simplicity we consider $J$ to be even.
We expect that in the $\bP\mu$-system this should correspond to
\beq
	\mu_{24} = \sum_{n=-\infty}^\infty C_n e^{2\pi n u}.
\eeq
We no longer expect $\mu_{24}$ to be either even or odd, since in the Bethe ansatz description of the state with generic mode numbers and filling fractions the Bethe roots are not distributed symmetrically.
As an example, for the ground state twist operator we have $\mu_{24}=\sinh(2\pi u)$, which is reproduced by choosing $C_{-1} = -1/2, C_1 = 1/2$ and all other $C$'s set to $0$.

It is straightforward to solve the $\pmu$-system in the same way as in section \ref{sec:slope_pmu_solve}, and we find the energy
\beq
	\gamma = \frac{\sqrt{\lambda}}{J} \frac{\sum_n C_n I_{J+1}(n \sqrt{\lambda}) }{\sum_n C_n I_{J}(n \sqrt{\lambda})/n} \, S,
\eeq
which can also be written in a more familiar form as
\beq
	\gamma = \sum_n \alpha_n \frac{n\sqrt{\lambda}}{J} \frac{I_{J+1}(n \sqrt{\lambda})}{I_{J}(n \sqrt{\lambda})} \, S,
\eeq
where
\beq
	\alpha_n = \frac{C_n I_{J}(n \sqrt{\lambda}) / n}{\sum_m C_m I_{J}(m \sqrt{\lambda}) / m}
\eeq
are the filling fractions.
The coefficients $C_n$ are additionally constrained by
\beq
	\sum_n C_n I_J(n\sqrt{\lambda}) = 0,
\eeq
which ensures that the $\bP_a$ functions have correct asymptotics. This constraint implies a relation between the filling fractions,
\beq
	\sum_n n \, \alpha_n = 0,
\eeq
which is also familiar from the asymptotic Bethe ansatz.


\section{Curvature function: details}
\label{sec:NLOapp}

In this appendix we will provide more details on the solution of the $\bP\mu$-system and calculation of curvature function for $J=2,3,4$ which was presented in the main text in section \ref{sec:curvature}.

\subsection{Corrections to $\mu_{ab}$ for $J=2$}
\label{sec:appmu2}
Here we present some details of the calculation of next to leading order corrections to $\mu_{ab}$ for $J=2$ omitted in the main text. 
As described in section \ref{sec:muNLOL2}, $\mu^{(1)}_{ab}$ are found as solutions of \eqref{eq:mudiscNLO} with appropriate asymptotics. 
The general solution of this equation consists of a general solution of the corresponding homogeneous equation (which can be reduced to one-parametric form \eqref{eq:periodicpart}) and a particular solution of the inhomogeneous one. The latter can be taken to be
\beq
\mu_{ab}^{disc}=\Sigma\cdot\(\bP_a^{(1)} \tilde\bP_b^{(1)}- \bP_b^{(1)} \tilde\bP_a^{(1)}\).
\eeq
One can get rid of the operation $\Sigma$, expressing $\mu_{ab}^{disc}$ in terms of $\Gamma'$ and $p_a'$. 
This procedure is based on two facts: the definition \eqref{paprime} of $p'_a$ and the statement that on functions decaying at infinity $\Sigma$ coincides with $\Gamma'$ defined by \eqref{Gammaprime}. 
After a straightforward but long calculation we find
\beqa
\mu_{31}^{disc} &=& \epsilon^2\Sigma\(\frac{1}{x^2}-x^2\)=
-\epsilon^2\;\;\(\Gamma\cdot x^2+p_2\),\\
\mu_{41}^{disc} &=&
   \epsilon^2\[-2I_1p_1-4I_1\Gamma\cdot x+\sinh(2\pi u)\(\Gamma\cdot x^2+p_0\)+
   \Gamma\cdot\sinh_-\(x-\frac{1}{x}\)^2\],\qquad\;\;\\
\mu_{43}^{disc} &=&
   -2{\epsilon^2}\[-2I_1p_1-4 I_1\Gamma\cdot x+\sinh(2\pi u)(p_2-p_0)+\Gamma\cdot\sinh_-\(x-\frac{1}{x}\)^2
   \],\\
\mu_{21}^{disc} &=&
   \epsilon^2\[2 I_1\Gamma\cdot x-\sinh(2\pi u)\;\Gamma\cdot x^2-\Gamma\cdot\sinh_-\(x^2+\frac{1}{x^2}\)
   \],\\
\mu_{24}^{disc}&=&
   \epsilon^2\[2I_1 \Gamma\cdot \sinh_- \(x+\frac{1}{x}\)+I_1^2p_0+ \right.\\
 && \left. +\sinh(2\pi u)\Gamma\cdot\sinh_-\(x^2-\frac{1}{x^2}\)-\Gamma\cdot \sinh_-^2\(x^2-\frac{1}{x^2}\)
   \] \nn.
\eeqa
Here we write $\Gamma$ and $p_a$ instead of $\Gamma'$ and $p'_a$ taking into account the discussion between equations \eqref{eq:ra} - \eqref{pa}.

\subsection{Solution of the $\bP\mu$-system for $J=3$}
\label{sec:appnlo3}

The particular solution of the inhomogeneous equation \eqref{eq:mudiscNLO} which we construct as $\mu_{31}^{disc}=\Sigma\cdot\(\bP_{a}^{(1)}\tilde\bP_{b}^{(1)}-\bP_{b}^{(1)}\tilde\bP_{a}^{(1)}\)$ can be written using the operation $\Gamma$  and $p_a$ defined by \eqref{pa} and \eqref{Gamma} as
\beqa
\mu_{31}^{disc} &=& \Sigma\cdot({\bf P}_3 \tilde{\bf P}_1-{\bf P}_1 \tilde{\bf P}_3)=-2\epsilon^2\[\Gamma x^3+p_3\],\\
\mu_{41}^{disc} &=&
   -\epsilon^2\[2p_2I_2+2I_2\Gamma x^2+2\Gamma\cdot\cosh_-+(I_0-\cosh(2\pi u))p_0\],\\
\mu_{34}^{disc} &=&
   {\epsilon^2}\[2I_2\Gamma x +I_0\Gamma x^3-\Gamma\cdot(x^3+x^{-3})\cosh_-+\cosh(2\pi u)(2p_3+\Gamma x^3)\], \qquad\\
\mu_{21}^{disc} &=&
   \epsilon^2\[2I_2\Gamma x+(I_0-\cosh(2\pi u))\Gamma x^3-\Gamma\((x^3+x^{-3})\cosh(2\pi u)\)\],\\
\mu_{24}^{disc} &=&
   -2\epsilon^2\[-\frac{1}{2}\Gamma\cdot\cosh_-^2\(x^3-x^{-3}\)+\(\frac{\cosh(2\pi u)}{2}-I_0\)\Gamma\cdot\frac{\cosh_-}{x^3}\right. \\
   &&-I_2\Gamma\cdot\(x+\frac{1}{x}\)\cosh_--\frac{1}{2}\cosh(2\pi u)\Gamma\cdot x^3\cosh_-+ \nn \\
   &&+\left.\frac{I_0}{2}\(I_0-\cosh(2\pi u)\)\Gamma\cdot x^3+\frac{I_1 I_2}{2\pi g}\Gamma x-I_2^2 p_1 \] \nn.
\eeqa
Alternatively one can use $p_a'$ and $\Gamma'$ instead of $p_a$ and $\Gamma$, as explained in the discussion between the equations \eqref{eq:ra} - \eqref{pa}.
The zero mode of the system \eqref{eq:P1L3} - \eqref{eq:P4L3}, which we added to the solution in equations \eq{eq:P1J3} - \eq{eq:P4J3} to ensure correct asymptotics, is given by
\beqa
\label{P1J3zm}
\bP_1^{\text{zm}}&=&L_1 x^{-1/2}+L_3x^{1/2}, \nn \\
\nn \bP_2^{\text{zm}}&=&-L_1 x^{1/2}\ch_-+L_2 x^{-1/2}-L_3x^{-1/2}\(\ch_-+\ofrac{2}I_0\)+L_4\(x^{1/2}-x^{-1/2}\),\\ \nn
\bP_3^{\text{zm}}&=&-L_1 x^{1/2}-L_3x^{1/2},\\ \nn
\bP_4^{\text{zm}}&=&-L_1\(I_0 x^{-1/2}+x^{-1/2}\cosh_-\)-L_2x^{1/2}+L_4(x^{1/2}-x^{-1/2})
\\ 
&-&L_3x^{1/2}\(\ch_-+\ofrac{2}I_0\).
\label{P4J3zm}
\eeqa


\subsection{Solution of the $\bP\mu$-system for $J=4$}
\label{sec:appL4}

Solution of the $\bP\mu$-system at next to leading order for $J=4$ is completely analogous to the case of $J=2$. 
The starting point is the leading order solution \eqref{eq:P1solLOevenL} - \eqref{eq:P4solLOevenL}. 
As described in section \ref{sec:appmu2}, from leading order $\bP_a$ we can find $\mu_{ab}$ at the next order. 
Its discontinuous part is
\label{sec:appnlo4}
\beqa
\mu_{31}^{disc} &=& -\epsilon^2\;\;\(\Gamma\cdot x^4+p_4\),\\
\mu_{41}^{disc} &=& \frac{1}{2} \epsilon ^2 \left(\sinh (2 \pi   u) \left(p_0+\Gamma \cdot x^4\right)+2 \left(I_1 p_1+I_3 p_3\right)+ \right.  \\ 
&& \left.+\Gamma \cdot \sinh_- \left(x^2-\frac{1}{x^2}\right)^2-2 \left(I_1+I_3\right) \left(\Gamma \cdot x^3+\Gamma \cdot x\right)\right), \nn \\
\mu_{43}^{disc} &=& \epsilon ^2 \left(\left(p_4-p_0\right) \sinh (2 \pi  u)+2 \left(I_1 p_1+I_3 p_3\right)- \right. \\ 
&&\left.-\Gamma \cdot\sinh_- \left(x^2-\frac{1}{x^2}\right)^2 +2 \left(I_1+I_3\right) \left(\Gamma \cdot x^3+\Gamma \cdot x\right)\right.\Bigg), \nn \\
\mu_{21}^{disc} &=& \epsilon ^2 \left(-\frac{1}{2} \sinh (2 \pi  u) \Gamma \cdot x^4+I_1 p_3+I_3 p_1-\right.  \\ 
&&\left. -\frac{1}{2} \Gamma \cdot \sinh_- \left(x^4+\frac{1}{x^4}\right)+I_1 \Gamma \cdot x^3+I_3 \Gamma \cdot x\right), \nn \\
\mu_{24}^{disc} &=& \epsilon ^2 \left(\frac{1}{2} \sinh (2 \pi u) \Gamma \cdot\sinh_- \left(x^4-\frac{1}{x^4}\right)+I_3^2 p_2+I_1 I_3 p_0- \right. \\ 
&& \left. -\frac{1}{2} \Gamma \cdot \sinh_-^2 \left(x^4-\frac{1}{x^4}\right) +I_1 \Gamma \cdot \sinh_- \left(x^3+\frac{1}{x^3}\right)+ \right. \nn \\ 
&& \left.+I_3 \Gamma \cdot\sinh_- \left(x+\frac{1}{x}\right)+\left(I_3^2-I_1^2\right)\Gamma \cdot x^2\right), \nn
\eeqa
and as discussed for $J=2$ the zero mode can be brought to the form
\beqa
&& \pi_{12}=0,\ \pi_{13}=0,\ \pi_{14}=0,\\
 && \pi_{24}=c_{1,24}\cosh{2\pi u},\ \pi_{34}=0.
 \label{eq:periodicpartL4}
 \eeqa
After that, we calculate $r_a$ by formula \eqref{eq:ra} and solve the expanded to next to leading order $\bP\mu$-system for $\bP_a^{(1)}$ as
\beqa
&&\bP^{(1)}_3=H \cdot r_3, \\
&&\bP^{(1)}_1=\frac{1}{2}\bP^{(1)}_3+K\cdot \(r_1-\frac{1}{2}r_3\),\\
&&\bP^{(1)}_4=K\cdot\[(H\cdot r_3)\sinh(2\pi u)+r_4-\frac{1}{2}r_3\sinh(2\pi u) \]-C(x+1/x),\\
&&\bP^{(1)}_2=H\cdot\[-\bP^{(1)}_4-\bP^{(1)}_1\sinh(2\pi u)+r_2\]+C/x,
\eeqa
where $C$ is a constant which is fixed by requiring correct asymptotics of $\bP_2$.
Finally we find leading coefficients $A_a$ of $\bP^{(1)}_a$ and use expanded up to ${\cal O}(S^2)$ formulas \eqref{AA1}, \eqref{AA2} in the same way as in section \ref{sec:resultL2} to obtain the result \eqref{gamma2L4}.

 \subsection{Result for $J=4$}
 \label{sec:SolvingPmuL4}
The final result for the curvature function at $J=4$ reads
\beqa
\footnotesize
\label{gamma2L4}
&&\gamma^{(2)}_{J=4}=\oint \frac{du_x}{2\pi i}\oint \frac{du_y}{2\pi i}
\frac{1}{i g^2(I_3-I_5)^3} \left.\Bigg[ \right.\\ && \left.\nn
\frac{2 \(\sh_-^x\)^2 y^4 \left(I_3 \left(x^{10}+1\right)-I_5 x^2 \left(x^6+1\right)\right)}{x^4 \left(x^2-1\right)}-\frac{2 \(\sh_-^y\)^2 x^4 \left(y^8-1\right) \left(I_3 x^2-I_5\right)}{\left(x^2-1\right) y^4}+ \right.\\
&&+\frac{4 \sh_-^x \sh_-^y \left(x^4 y^4-1\right) \left(I_3+I_3 x^6 y^4-I_5 x^2 \left(x^2 y^4+1\right)\right)}{x^4 \left(x^2-1\right) y^4}\nn\\
&&+\left.\sh_-^y\(
\left(y^4+y^{-4}\right) x^{-1}\left(\left(I_1 I_5-I_3^2\right) \left(3 x^4+1\right)-2 I_1 I_3 x^6\right)+
\right.\right.\nn\\&&\left.\left.
+\frac{2 I_3 x^2
   \left(I_5 \left(x^2+1\right) x^2+I_1 \left(1-x^2\right)\right)-I_1 I_5 \left(x^2-1\right)^2+I_3^2 \left(-2 x^6+x^4+1\right)}{x(x^2-1)}+
   \right.\right.\nn\\&&\left.\left.
   +2
   \left(y^3+y^{-3}\right) \frac{I_1 I_3 x^6-I_1 I_5 x^4-I_3^2 \left(x^2-1\right)}{x^2-1}-
   \right.\right.\nn\\&&\left.\left.
   -2 I_3 \left(y+y^{-1}\right)
   \frac{ I_1\left(x^2-1\right)-I_3 \left(x^6-x^2+1\right)+I_5 \left(x^4-x^2+1\right)}{x^2-1}
   \)+
       \right.\nn\\&&\left.
   +\frac{4 x^6 y^2 I_3 \left(I_3^2-I_1^2\right)}{x^2-1}+\frac{4 x y I_1 \left(I_3 y^2+I_1\right) \left(I_3+I_5\right)}{x^2-1}+
   \right.\nn\\&&\left.
  \frac{2 y^4 \left(I_1+I_3\right) \left(I_1 I_5-I_3^2\right)}{x^2-1}
-\frac{2 y \left(y^2+1\right) \left(I_1+I_3\right) \left(I_1 I_5-I_3^2\right)}{x \left(x^2-1\right)}-
    \right.\nn\\&&\left.
-\frac{2 x^3 y \left(I_1+I_3\right) \left(I_1 \left(2 I_3+\left(3 \
y^2+1\right) I_5\right)-I_3 \left(2 I_5 y^2+\left(y^2+3\right) I_3\right)\right)}{x^2-1}
    \right.\nn\\&&\left.+
\frac{2 x^2 y^4 \left(-I_3^3-I_1 \left(3 I_3+I_5\right) I_3+I_1^2 I_5\right)}{x^2-1} +\right. \nn \\
&& \left.+
\frac{2 x^4 y \left(I_1^2 \left(2 y I_5-2 y^3 I_3\right)-2 y \left(y^2+1\right) I_3^2 I_5\right)}{x^2-1}+
    \right.\nn\\&&\left.
 +\frac{4 x^5 y I_3 \left(2 I_1^2 y^2+I_3 \left(I_5-I_3\right) y^2+I_1 \left(I_3+I_5\right)\right)}{x^2-1}
  \right] \frac{1}{4\pi i}\d_u \log\frac{\Gamma (i u_x-i u_y+1)}{\Gamma (1-i u_x+i u_y)}\nn
\eeqa
\normalsize
where, similarly to $J=2,3$, the integrals go around the branch between $-2g$ and $2g$.


\newpage
\subsection{Weak coupling expansion}
\label{sec:weakS3}

First, we give the expansion of our results for the curvature functions $\gamma_J^{(2)}$ to 10 loops. 
We start with $J=2$:
\beqa
\label{weak22long}
	\gamma_{J=2}^{(2)}&=&-8 g^2 \zeta_3+g^4 \left(140 \zeta_5-\frac{32 \pi ^2 \zeta_3}{3}\right)+g^6
   \left(200 \pi ^2 \zeta_5-2016 \zeta_7\right)
	\\ \nn
	&+&g^8 \left(-\frac{16 \pi ^6 \zeta_3}{45}-\frac{88 \pi ^4 \zeta_5}{9}-\frac{9296 \pi ^2 \zeta_7}{3}+27720 \zeta_9\right)
	\\ \nn
	&+&g^{10} \left(\frac{208 \pi ^8 \zeta_3}{405}+\frac{160 \pi ^6 \zeta_5}{27}+144 \pi ^4 \zeta_7+45440 \pi ^2 \zeta_9-377520 \zeta_{11}\right)
	\\ \nn
	&+&g^{12}
   \left(-\frac{7904 \pi ^{10} \zeta_3}{14175}-\frac{17296 \pi ^8 \zeta_5}{4725}-\frac{128 \pi ^6 \zeta_7}{15}-\frac{6312 \pi ^4 \zeta_9}{5}
	\right.
	\\ \nn
	&&\Bigl.\ \ \ \ \ \ \
	-653400 \pi
   ^2 \zeta_{11}+5153148 \zeta_{13}\Bigr)
	\\ \nn
	&+&g^{14} \Bigl(\frac{1504 \pi ^{12} \zeta_3}{2835}+\frac{106576 \pi ^{10} \zeta_5}{42525}-\frac{18992 \pi ^8 \zeta_7}{405}
-\frac{16976 \pi ^6 \zeta_9}{15}
	\Bigr.
	\\ \nn
	&& \Bigl. \ \ \ \ \ \ \
	+\frac{25696 \pi ^4 \zeta_{11}}{9}+\frac{28003976 \pi ^2 \zeta_{13}}{3}-70790720 \zeta_{15}\Bigr)
	\\ \nn
	&+&g^{16}
   \Bigl(-\frac{178112 \pi ^{14} \zeta_3}{382725}-\frac{239488 \pi ^{12} \zeta_5}{127575}+\frac{2604416 \pi ^{10} \zeta_7}{42525}+\frac{8871152 \pi ^8 \zeta_9}{4725}
		\Bigr.
	\\ \nn
	&& \Bigl. \ \ \ \ \ \ \
	+\frac{30157072 \pi ^6 \zeta_{11}}{945}+\frac{8224216 \pi ^4 \zeta_{13}}{45}-133253120 \pi ^2 \zeta_{15}
			\Bigr.
	\\ \nn
	&& \Bigl. \ \ \ \ \ \ \
	+979945824 \zeta_{17}\Bigr)
	\\ \nn
	&+&g^{18}
   \Bigl(\frac{147712 \pi ^{16} \zeta_3}{382725}+\frac{940672 \pi ^{14} \zeta_5}{637875}-\frac{490528 \pi ^{12} \zeta_7}{8505}-\frac{358016 \pi ^{10} \zeta_9}{189}
	\Bigr.
	\\ \nn
	&& \Bigl. \ \ \ \ \ \ \
	-\frac{37441312 \pi ^8 \zeta_{11}}{945}-\frac{9616256 \pi ^6 \zeta_{13}}{15}-\frac{16988608 \pi ^4 \zeta_{15}}{3}
		\Bigr.
	\\ \nn
	&& \Bigl. \ \ \ \ \ \ \
	+1905790848 \pi ^2 \zeta_{17}-13671272160 \zeta_{19}\Bigr)
	\\ \nn
	&+&g^{20} \Bigl(-\frac{135748672 \pi ^{18} \zeta_3}{442047375}-\frac{103683872 \pi ^{16} \zeta_5}{88409475}+\frac{1408423616 \pi
   ^{14} \zeta_7}{29469825}
			\Bigr.
	\\ \nn
	&& \Bigl. \ \ \ \ \ \ \
	+\frac{2288692288 \pi ^{12} \zeta_9}{1403325}+\frac{34713664 \pi ^{10} \zeta_{11}}{945}+\frac{73329568 \pi ^8 \zeta_{13}}{105}
				\Bigr.
	\\ \nn
	&& \Bigl. \ \ \ \ \ \ \
	+\frac{305679296 \pi ^6 \zeta_{15}}{27}+121666688 \pi ^4 \zeta_{17}-27342544320 \pi ^2 \zeta_{19}
					\Bigr.
	\\ \nn
	&& \Bigl. \ \ \ \ \ \ \
	+192157325360 \zeta_{21}\Bigr).
\eeqa
Next for $J=3$ we find,
\beqa
\label{weak23long}
\gamma_{J=3}^{(2)}&=&-2 g^2 \zeta_3+g^4 \left(12 \zeta_5-\frac{4 \pi ^2 \zeta_3}{3}\right)+g^6
   \left(\frac{2 \pi ^4 \zeta_3}{45}+8 \pi ^2 \zeta_5-28 \zeta_7\right)
	\\ \nn
	&+&g^8\;
   \left(-\frac{4 \pi ^6 \zeta_3}{45}-\frac{4 \pi ^4 \zeta_5}{15}-528 \zeta_9\right)
	\\ \nn
	&+&g^{10} \left(\frac{934 \pi ^8 \zeta_3}{14175}+\frac{8 \pi ^6 \zeta_5}{9}-\frac{82 \pi ^4 \zeta_7}{9}-900 \pi ^2 \zeta_9+12870 \zeta_{11}\right)
	\\ \nn
	&+&g^{12} \left(-\frac{572 \pi ^{10} \zeta_3}{14175}-\frac{104 \pi ^8
   \zeta_5}{175}-\frac{256 \pi ^6 \zeta_7}{45}+\frac{2476 \pi ^4 \zeta_9}{9}
	\right. \\ \nn
	&&\ \ \ \ \ \ \ \left.+\frac{57860 \pi ^2 \zeta_{11}}{3}-208208 \zeta_{13}\right)
	\\ \nn
	&+&g^{14}
   \left(\frac{2878 \pi ^{12} \zeta_3}{127575}+\frac{404 \pi ^{10} \zeta_5}{1215}+\frac{326 \pi ^8 \zeta_7}{75}+\frac{3352 \pi ^6 \zeta_9}{135}
	\right. \\ \nn
	&&\ \ \ \ \ \ \ -\left.\frac{80806 \pi ^4 \zeta_{11}}{15}
	-316316 \pi ^2 \zeta_{13}+2994992 \zeta_{15}\right)
	\\ \nn
	&+&g^{16} \left(-\frac{159604 \pi ^{14} \zeta_3}{13395375}-\frac{257204
   \pi ^{12} \zeta_5}{1488375}-\frac{14836 \pi ^{10} \zeta_7}{6075}-\frac{71552 \pi
   ^8 \zeta_9}{2025}
	\right.
	\\ \nn
	&&\ \ \ \ \ \ \  \left.+\frac{4948 \pi ^6 \zeta_{11}}{189}+\frac{4163068 \pi ^4 \zeta_{13}}{45}+\frac{14129024 \pi ^2 \zeta_{15}}{3}-41116608 \zeta_{17}\right)
		\\ \nn
	&+&g^{18}
   \left(\frac{494954 \pi ^{16} \zeta_3}{81860625}+\frac{156368 \pi ^{14} \zeta_5}{1819125}+\frac{6796474 \pi ^{12} \zeta_7}{5457375}+\frac{332 \pi ^{10} \zeta_9}{15}
		\right.
	\\ \nn
	&&\ \ \ \ \ \ \ \left.+\frac{1745318 \pi ^8 \zeta_{11}}{4725} -\frac{868088 \pi ^6 \zeta_{13}}{315}-\frac{22594208 \pi ^4 \zeta_{15}}{15}
	\right. \\ \nn
	&&\ \ \ \ \ \ \ \Bigl.-67084992 \pi ^2 \zeta_{17}+553361016
   \zeta_{19}\Biggr)
	\\ \nn
	&+&g^{20} \left(-\frac{940132 \pi ^{18} \zeta_3}{315748125}-\frac{244456 \pi ^{16} \zeta_5}{5893965}-\frac{29637008 \pi ^{14}
   \zeta_7}{49116375}-\frac{11808196 \pi ^{12} \zeta_9}{1002375}
	\right.
	\\ \nn
		&&\ \ \ \ \ \ \ -\left.\frac{2265364 \pi
   ^{10} \zeta_{11}}{8505}-\frac{68767984 \pi ^8 \zeta_{13}}{14175}+\frac{480208 \pi ^6
   \zeta_{15}}{9}
	\right.
	\\ \nn
		&&\ \ \ \ \ \ \ +\left.
	\frac{71785288 \pi ^4 \zeta_{17}}{3}+934787840 \pi ^2 \zeta_{19}-7390666360 \zeta_{21}\right).
\eeqa
Finally, for $J=4$,
\beqa
\label{weak24long}
	\gamma_{J=4}^{(2)}&=&
	g^2 \left(-\frac{14 \zeta_3}{5}+\frac{48 \zeta_5}{\pi ^2}-\frac{252 \zeta_7}{\pi ^4}\right)
	\\ \nn
	&+&g^4
   \left(-\frac{22 \pi ^2 \zeta_3}{25}+\frac{474 \zeta_5}{5}-\frac{8568 \zeta_7}{5 \pi ^2}+\frac{8316
   \zeta_9}{\pi ^4}\right)
\\ \nn
&+&g^6 \left(\frac{32 \pi ^4 \zeta_3}{875}+\frac{3656 \pi ^2 \zeta_5}{175}-\frac{56568 \zeta_7}{25}+\frac{196128 \zeta_9}{5 \pi ^2}-\frac{185328 \zeta_{11}}{\pi
   ^4}\right)
\\ \nn
&+&g^8 \Bigl(-\frac{4 \pi ^6 \zeta_3}{175}-\frac{68 \pi ^4 \zeta_5}{75}-\frac{55312 \pi ^2
   \zeta_7}{125}+\frac{1113396 \zeta_9}{25}-\frac{3763188 \zeta_{11}}{5 \pi ^2}
	\Bigr.
	\\ \nn
	&& \Bigl. \ \ \ \ \ \ \
+\frac{3513510 \zeta_{13}}{\pi ^4}\Bigr)
\\ \nn
&+&g^{10} \Bigl(\frac{176 \pi ^8 \zeta_3}{16875}+\frac{2488 \pi ^6 \zeta_5}{7875}+\frac{2448 \pi ^4 \zeta_7}{125}+\frac{209532 \pi ^2 \zeta_9}{25}-\frac{3969878 \zeta_{11}}{5}
	\Bigr.
	\\ \nn
	&& \Bigl. \ \ \ \ \ \ \
+\frac{13213200 \zeta_{13}}{\pi ^2}-\frac{61261200 \zeta_{15}}{\pi ^4}\Bigr)
\\ \nn
&+&g^{12}
   \Bigl(-\frac{88072 \pi ^{10} \zeta_3}{20671875}-\frac{449816 \pi ^8 \zeta_5}{4134375}-\frac{327212 \pi
   ^6 \zeta_7}{65625}-\frac{338536 \pi ^4 \zeta_9}{875}
	\Bigr.
	\\ \nn
	&& \Bigl. \ \ \ \ \ \ \
-\frac{129520798 \pi ^2 \zeta_{11}}{875}+\frac{66969474 \zeta_{13}}{5}-\frac{220540320 \zeta_{15}}{\pi ^2}
	\Bigr.
	\\ \nn
	&& \Bigl. \ \ \ \ \ \ \
+\frac{1017636048 \zeta_{17}}{\pi ^4}\Bigr)
\\ \nn
&+&g^{14} \Bigl(\frac{795136 \pi ^{12} \zeta_3}{487265625}+\frac{522784 \pi ^{10} \zeta_5}{13921875}+\frac{4021288 \pi ^8 \zeta_7}{2953125}+\frac{1869152 \pi ^6 \zeta_9}{21875}
	\Bigr.
	\\ \nn
	&& \Bigl. \ \ \ \ \ \ \
+\frac{18573952 \pi ^4 \zeta_{11}}{2625}+\frac{62633272 \pi ^2 \zeta_{13}}{25}-\frac{1092799344
   \zeta_{15}}{5}
	\Bigr.
	\\ \nn
	&& \Bigl. \ \ \ \ \ \ \
+\frac{17844607872 \zeta_{17}}{5 \pi ^2}-\frac{16405526592 \zeta_{19}}{\pi ^4}\Bigr)
\\ \nn
&+&g^{16}
   \Bigl(-\frac{30581888 \pi ^{14} \zeta_3}{51162890625}-\frac{43988768 \pi ^{12} \zeta_5}{3410859375}-\frac{446380184 \pi ^{10} \zeta_7}{1136953125}
	\Bigr.
	\\ \nn
	&& \Bigl. \ \ \ \ \ \ \
-\frac{20108936 \pi ^8 \zeta_9}{984375}
-\frac{31755036 \pi ^6 \zeta_{11}}{21875}-\frac{321449336 \pi ^4 \zeta_{13}}{2625}
	\Bigr.
	\\ \nn
	&& \Bigl. \ \ \ \ \ \ \
-\frac{1031925232 \pi ^2 \zeta_{15}}{25}
	+\frac{87296960712 \zeta_{17}}{25}-\frac{283092985656
   \zeta_{19}}{5 \pi ^2}
	\Bigr.
	\\ \nn
	&& \Bigl. \ \ \ \ \ \ \
+\frac{259412389236 \zeta_{21}}{\pi ^4}\Bigr)
\\ \nn
&+&g^{18} \Bigl(\frac{6706432 \pi ^{16}
   \zeta_3}{31672265625}+\frac{816838192 \pi ^{14} \zeta_5}{186232921875}+\frac{2004636572 \pi ^{12} \zeta_7}{17054296875}
	\Bigr.
	\\ \nn
	&& \Bigl. \ \ \ \ \ \ \
+\frac{1950592976 \pi ^{10} \zeta_9}{378984375}
+\frac{2220222512 \pi ^8 \zeta_{11}}{6890625}+\frac{20963856 \pi ^6 \zeta_{13}}{875}
	\Bigr.
	\\ \nn
	&& \Bigl. \ \ \ \ \ \ \
+\frac{254959316 \pi ^4 \zeta_{15}}{125}
+\frac{584553371616 \pi ^2 \zeta_{17}}{875}
	\Bigr.
	\\ \nn
	&& \Bigl. \ \ \ \ \ \ \
-\frac{1375388084412 \zeta_{19}}{25}+\frac{4432313039616 \zeta_{21}}{5 \pi ^2}
-\frac{4049650420200 \zeta_{23}}{\pi ^4}\Bigr)
\eeqa
\beqa
\nn
&+&g^{20}
   \Bigl(-\frac{15308976272 \pi ^{18} \zeta_3}{209512037109375}-\frac{1764947984 \pi ^{16} \zeta_5}{1197211640625}-\frac{18667123736 \pi ^{14} \zeta_7}{517313671875}
	\Bigr.
	\\ \nn
	&& \Bigl. \ \ \ \ \ \ \
-\frac{538293689008 \pi ^{12} \zeta_9}{399070546875}-\frac{657466372 \pi ^{10} \zeta_{11}}{8859375}-\frac{119709052 \pi ^8 \zeta_{13}}{23625}
	\Bigr.
	\\ \nn
	&& \Bigl. \ \ \ \ \ \ \
-\frac{9095498848 \pi ^6 \zeta_{15}}{23625}-\frac{260407748416 \pi ^4 \zeta_{17}}{7875}-\frac{1869110789976 \pi ^2 \zeta_{19}}{175}
	\Bigr.
	\\ \nn
	&& \Bigl. \ \ \ \ \ \ \
+\frac{4293062840352 \zeta_{21}}{5}-\frac{13755955395600 \zeta_{23}}{\pi ^2}+\frac{62673161265000 \zeta_{25}}{\pi
   ^4}\Bigr).
\eeqa
With the help of several Mathematica packages for dealing with harmonic sums (as noted in the main text), we have also computed the weak coupling expansion of the anomalous dimensions at order $S^3$, using the known predictions from ABA which are available for any spin at $J=2$ and $J=3$. 
For $J=2$ we have computed the expansion to three loops:
\beqa
	\gamma_{J=2}^{(3)}&=&g^2\frac{4}{45}\pi^4+g^4 \left(40 \zeta_3^2-\frac{28 \pi ^6}{405}\right)\\ \nn
	&+&g^6 \left(\frac{192}{5} \zeta_{5,3}-\frac{6992 \zeta_3
   \zeta_5}{5}+\frac{280 \pi ^2 \zeta_3^2}{3}+\frac{6962 \pi ^8}{212625}\right)+\mathcal{O}(g^8).
\eeqa
Compared to the $S^2$ part, a new feature is the appearance of multiple zeta values -- here we have $\zeta_{5,3}$, which is defined by
\begin{equation}
	\zeta_{a_1,a_2,\ldots,a_k}=\sum_{0< n_1<n_2<\ldots<n_k<\infty}
	\frac 1{n_{1}^{a_1}n_{2}^{a_2}\ldots n_k^{a_k}}\,
\end{equation}
and cannot be reduced to simple zeta values $\zeta_n$.
For $J=3$ we have obtained the expansion to four loops:
\beqa
\nn
	\gamma_{J=3}^{(3)}&=&\frac{1}{90}\pi^4g^2+g^4\(4 \zeta_3^2+\frac{\pi ^6}{1890}\)
	+g^6\(4\zeta_{5,3}+4 \pi ^2 \zeta_3^2-72 \zeta_3 \zeta_5-\frac{2
   \pi ^8}{675}\)\\ \nn
   &+&g^8\left(-112 \zeta_{2,8}+\frac{20}{3} \pi ^2 \zeta_{5,3}+728 \zeta_3
   \zeta_7+448 \zeta_5^2-\frac{224}{3} \pi ^2 \zeta_3
   \zeta_5\right.\\
	&& \left.+\frac{4 \pi ^4 \zeta_3^2}{5}-\frac{41 \pi
   ^{10}}{133650}\right)
   +\mathcal{O}(g^{10}).
\eeqa
These results would be useful for verifying the next small spin anomalous dimension expansion coefficient after the curvature.


% \section{Higher mode numbers}
% \label{sec:appHigher}


% \subsection{Curvature function and higher mode numbers}
% \label{sec:appN}




% \section{Exact formulae for one-loop correction}\label{appA}
% \subsection{Main formula for one-loop correction and notations}
% In \cite{Gromov:2011de} a general formula was derived describing the one loop correction
% to the energy of the generic $(S,J,n)$ folded string solution.
% There are three contributions to one loop energy shift that are different by their nature.
% They can be separated into an ``anomaly" contribution, a contribution from the dressing phase
% and a wrapping contribution, which is missing in the ABA approach, but present in the
% Y-system
% \beq
% \Delta_{\rm1-loop}=\delta \Delta_{\rm anomaly} +\delta \Delta_{\rm dressing} + \delta \Delta_{\rm wrapping}\;,
% \eeq
% where each of these contributions is simply an integral of some closed form expression,
% \beqa
% \label{eq:delta_E_3}
  % \delta \Delta_{\rm anomaly} &=& -\frac{4}{ab-1}\int_a^b \frac{dx} {2\pi i}
  % \frac{y(x)}{x^2-1} \partial_x \log \sin p_{\hat 2}\;,\\
  % \label{eq:delta_E_1}
  % \delta \Delta_{\rm dressing} &=& \sum_{ij} (-1)^{F_{ij}} \int\limits_{-1}^{1} \frac{dz} {2\pi
    % i} \left( \Omega^{ij}(z) \, \partial_z \frac{i (p_i -p_j)}{2} \right)\;,\\
  % \label{eq:delta_E_2}
  % \delta \Delta_{\rm wrapping} &=& \sum_{ij} (-1)^{F_{ij}} \int\limits_{-1}^{1} \frac{dz} {2\pi
    % i} \left( \Omega^{ij}(z) \, \partial_z \log (1- e^{-i(p_i -p_j)}) \right)\;.
% \eeqa
% in this sum $i$ takes values $\hat 1,\hat 2,\tilde 1,\tilde 2$ whereas $j$ runs over
% $\hat 3,\hat 4,\tilde 3,\tilde 4$.

% Let us explain the notations. The quasi-momenta:
% \beqa
  % \label{eq:p_a}\nn
  % p_{\hat 2} &=& \pi n - 2\pi n{\cal J} \left( \frac{a}{a^2-1} -
    % \frac{x}{x^2-1} \right) \sqrt{\frac{(a^2-1)
      % (b^2-x^2)}{(b^2-1)(a^2-x^2)}} \\ \nn &+& \frac{8\pi n a b {\cal S} F_1(x)}{(b-a)(ab+1)} +
  % \frac{2\pi n {\cal J} (a-b) F_2(x)}{\sqrt{(a^2-1)(b^2-1)}},\\
  % \label{eq:p_s}
  % p_{\tilde 2} &=& \frac{2\pi {\cal J}x}{x^2-1}.
% \eeqa
% The integer $n$ (the mode number) is related to the number of spikes.
% All the other quasi-momenta can be
% found from
% \beqa
  % \label{eq:quasimomenta_symmetry_A}
  % p_{\hat{2}} (x) &=& -p_{\hat{3}}(x) = -p_{\hat{1}}(1/x) = p_{\hat{4}}
  % (1/x)\;,\\
  % \label{eq:quasimomenta_symmetry_S}
  % p_{\tilde{2}} (x) &=& -p_{\tilde{3}} (x) = p_{\tilde{1}} (x) =
  % -p_{\tilde{4}}(x)\;.
% \eeqa
% The functions $F_1(x)$ and $F_2(x)$ can be expressed
% in terms of the elliptic integrals:
% \begin{eqnarray}
 % \label{eq:not}
 % F_1(x) &=& i F \left( i \sinh^{-1}
 % \sqrt{\frac{(b-a)(a-x)}{(b+a)(a+x)}} | \frac{(a+b)^2}{(a-b)^2} \nonumber
 % \right)\;, \\
% \nn
 % F_2(x) &=& i E \left( i \sinh^{-1}
 % \sqrt{\frac{(b-a)(a-x)}{(b+a)(a+x)}} | \frac{(a+b)^2}{(a-b)^2}
 % \right)\;.
 % \end{eqnarray}
% Finally the off-shell fluctuation energies are
% \begin{eqnarray}\nn
  % \label{eq:frequencies}
  % \Omega^{\hat{1}\hat{4}} (x) &=& -\Omega^{\hat{2}\hat{3}}(1/x) -2\;,\;\;\\
  % \Omega^{\hat{1}\hat{3}} (x) &=& \Omega^{\hat{2} \hat{4}} (x) = \frac12
  % \Omega^{\hat{1}\hat{4}}(x) + \frac12 \Omega^{\hat{2}\hat{3}}(x)\;,\nn\\
  % \Omega^{\hat{1}
    % \tilde{3}}(x)
% &=&
  % \Omega^{\hat{1}\tilde{4}}(x)
    % =  \Omega^{\hat{4}\tilde{1}}(x)
    % =  \Omega^{\hat{4}\tilde{2}}(x)
     % =
  % \frac12\, \Omega^{\tilde{2}\tilde{3}} (x) + \frac12\,\Omega^{\hat{1}\hat{4}}(x),\\
  % \Omega^{\hat{2}\tilde{3}}(x)
% &=&
  % \Omega^{\hat{2}\tilde{4}}(x)
% =
  % \Omega^{\tilde{1}\hat{3}}(x)
% =
  % \Omega^{\tilde{2}\hat{3}}(x)
% = \frac12\, \Omega^{\tilde{2}\tilde{3}}(x) + \frac12\, \Omega^{\hat{2}\hat{3}}(x),\nn\\
  % \Omega^{\tilde{2}\tilde{3}}(x)
% &=&
  % \Omega^{\tilde{2}\tilde{4}}(x)
% =
  % \Omega^{\tilde{1}\tilde{3}}(x)
% =
  % \Omega^{\tilde{1}\tilde{4}}(x)\;,\nn
% \end{eqnarray}
% where
% \beqa
% \label{oma}
% \Omega^{\tilde 2\tilde 3}(x)&=&\frac{2}{a
  % b-1}\frac{\sqrt{a^2-1}\sqrt{b^2-1}}{x^2-1}\;,\\
% \label{oms}
% \Omega^{\hat 2\hat 3}(x)&=&\frac{2}{a b-1}\(1-\frac{y(x)}{x^2-1}\)\;.
% \eeqa
% and $y(x)=\sqrt{x-a} \sqrt{a+x} \sqrt{x-b} \sqrt{b+x}$.

% In the small $\cal{S}$ limit these expressions can be expanded,
% \beqa\label{anomaly_dressing}
	% &&\delta\Delta_{anomaly} = \frac{-1}{2( \mathcal{J}^3 + \mathcal{J})} \, \mathcal{S}
	% + \left[ \frac{2 \, \mathcal{J}^4 + 15 \, \mathcal{J}^2 + 4}{16 \, \mathcal{J}^3(\mathcal{J}^2 + 1)^{5/2}} - \frac{\pi^2 n^2}{12 \, \mathcal{J}^3 \, \sqrt{\mathcal{J}^2 + 1}} \right] \mathcal{S}^2   \\
	% &&+ \left[ \frac{3 \, \mathcal{J}^8 -32 \, \mathcal{J}^6 - 146 \, \mathcal{J}^4 - 68 \, \mathcal{J}^2 - 16}{64 \, \mathcal{J}^5 (1+ \, \mathcal{J}^2)^4} + \frac{\pi^2 n^2 (\, \mathcal{J}^4 + 4 \, \mathcal{J}^2 + 2)}{24 \, \mathcal{J}^5 (1 + \, \mathcal{J}^2)^2} + \frac{\pi^4 n^4}{180 \, \mathcal{J}^5} \right] \mathcal{S}^3 \nonumber
	% + \mathcal{O}(\mathcal{S}^4)  \\
	% &&\delta\Delta_{dressing} = \left[ \frac{n \, (\mathcal{J}^2 + 2) \, \mathrm{coth}^{-1} (\sqrt{\, \mathcal{J}^2 + 1} + \, \mathcal{J})}{\, \mathcal{J}^3(\, \mathcal{J}^2 + 1)^{3/2}} -\frac{n}{2 \, \mathcal{J}^3 (\, \mathcal{J}^2 + 1)}  \right] \mathcal{S}^2 \nonumber \\
	% &&+ \left[ -\frac{n(3 \, \mathcal{J}^6 + 13 \, \mathcal{J}^4 + 22 \, \mathcal{J}^2 + 8)\, \mathrm{coth}^{-1} (\sqrt{\, \mathcal{J}^2 + 1} + \, \mathcal{J})}{2 \, \mathcal{J}^5 (1 + \, \mathcal{J}^2)^3} + \frac{n(9 \, \mathcal{J}^4 + 31 \, \mathcal{J}^2 + 10)}{12 \, \mathcal{J}^5 (1 + \, \mathcal{J}^2)^{5/2}} \right] \mathcal{S}^3 \nonumber
	% + \mathcal{O}(\mathcal{S}^4)
% \eeqa
% the expansion of the third integral $\delta\Delta_{wrapping}$ is more complicated, and we advice the reader to use
% the equation \eq{delta_oneloop} instead which includes all contributions. What we can, however, say is that
	% $\delta\Delta_{wrapping} = {\cal O}(e^{-2\pi{\cal J}})$
% and thus this term is irrelevant for the large ${\cal J}$ expansion.
% This makes the expressions \eq{anomaly_dressing} particularly convenient for
% small ${\cal S}$ followed by large ${\cal J}$ expansions,
% where as the exact ${\cal J}$ expression in \eq{delta_oneloop} is not
% very convenient since the sum of the expansion does not converge.


% \newpage
% \subsection{One Loop $(S,J)$ Folded String {\it Mathematica} {Code}}
% In order to fix all our conventions as well as for the convenience of the
% reader we include a simplified  {\it Mathematica} {code}
% we used to check our results numerically\\
% {\footnotesize
% \verb" "\\
% \verb"GS=((2*(a*b+1))*(b*EllipticE[1-a^2/b^2]-a*EllipticK[1-a^2/b^2]))/(4*Pi*a*b);"\\
% \verb"GJ=((4*Sqrt[(a^2-1)*(b^2-1)])*EllipticK[1-a^2/b^2])/(4*Pi*b);"\\
% \verb"y=Sqrt[x-a]*Sqrt[x+a]*Sqrt[x-b]*Sqrt[x+b];"\\
% \verb"F1[x_] =I*EllipticF[I*ArcSinh[Sqrt[-(((a-b)*(a-x))/((a+b)*(a+x)))]], (a+b)^2/(a-b)^2];"\\
% \verb"F2[x_] =I*EllipticE[I*ArcSinh[Sqrt[-(((a-b)*(a-x))/((a+b)*(a+x)))]], (a+b)^2/(a-b)^2];"\\
% \verb"pA[x_] =n*Pi-2*Pi*n*j*(a/(a^2-1)-x/(x^2-1))*Sqrt[((a^2-1)*(b^2-x^2))/((b^2-1)*(a^2-x^2))] +"\\
% \verb"       (8*a*b*s*Pi*n*F1[x])/((b-a)*(a*b+1))+(2*Pi*n*j*(a-b)*F2[x])/Sqrt[(a^2-1)*(b^2-1)];"\\
% \verb"pS[x_]=(2*Pi*n*j*x)/(x^2-1);"\\
% \verb"X[z_]=z+Sqrt[z^2-1];"\\
% \verb"OA[x_]=(2*(1-y/(x^2-1)))/(a*b-1);"\\
% \verb"OS[x_]=(2*(-(y /. x->1)))/((a*b-1)*(x^2-1));"\\
% \verb"ab[j_, s_] :=ab[j, s]=Chop[FindRoot[{s==GS, j==GJ}, {{b, Sqrt[j^2+1]+j+Sqrt[s]/10}"\\
% \verb"                                                    ,{a, Sqrt[j^2+1]+j-Sqrt[s]/10}}]];"\\
% \verb"OneLoop[jj_, ss_, nn_] := Block[{sb0=Join[ab[jj, ss], {j->jj, s->ss, n->nn}]},"\\
% \verb"tn0=(2*Im[pA[X[z]]-pS[X[z]]]*Im[D[OA[X[z]]-OS[X[z]], z]])/Pi /. sb0;"\\
% \verb"Edressing=NIntegrate[tn0, {z, 0, 1}];"\\
% \verb"tn1=(2*D[OS[X[z]], z]*Log[((1-Exp[(-I)*pS[X[z]]-I*pA[X[z]]])*(1-Exp[(-I)*pS[X[z]]+I*pA[1/X[z]]]))/"\\
% \verb"                                                              (1-Exp[-2*I*pS[X[z]]])^2])/Pi /. sb0;"\\
% \verb"tn2=-((2*D[OA[X[z]], z]*Log[((1-Exp[-2*I*pA[X[z]]])*(1-Exp[(-I)*pA[X[z]]+I*pA[1/X[z]]]))/"\\
% \verb"                                                  (1-Exp[(-I)*pS[X[z]]-I*pA[X[z]]])^2])/Pi) /. sb0;"\\
% \verb"Ewrapping=NIntegrate[Im[tn1+tn2], {z, 0, 1}];"\\
% \verb"tn=-((4*y*D[Log[Sin[pA[x]]], x])/((a*b-1)*(2*Pi*I)*(x^2-1))) /. sb0;"\\
% \verb"Eanomaly=Re[NIntegrate[tn, {x, a /. sb0, ((a+b)*(1+I))/(2*10) /. sb0, b /. sb0}]];"\\
% \verb"Edressing+Ewrapping+Eanomaly];"
% \verb" "\\
% }\\
% To compute $\Delta_{\rm 1-loop}$ simply run \verb"OneLoop["${\cal J},{\cal S}, n$\verb"]" in {\it Mathematica}.

% \section{$S^3$ and $S^4$ order}\label{AppS4}
% The $\mathcal{S}^3$ order term in the expression of \eq{delta_oneloop} is given by
% \beqa
% \label{delta_oneloop_3}
% \delta\Delta_{1-loop}^{(3)} &=&-\frac{6 \, \mathcal{J}^8 + 48 \, \mathcal{J}^6+ 138 \, \mathcal{J}^4 + 352 \, \mathcal{J}^2 + 117 }{64 \, \mathcal{J}^5 (\, \mathcal{J}^2 + 1)^4 }  \\
 % &+& \sum_{m>0,m\neq n}
% \nn \frac{P_3(n,m,{\cal J}) }{2 \, \mathcal{J}^5 (\, \mathcal{J}^2 + 1)^{3/2} (m^2 - n^2)^4  (\mathcal{J}^2 n^2 + m^2)^{5/2}}
% \eeqa
% and the $\mathcal{S}^4$ order term is given by
% \beqa
% \label{delta_oneloop_4}
% \delta\Delta_{1-loop}^{(4)} &=&
% \frac{45 {\cal J}^{12}+717 {\cal J}^{10}+3429 {\cal J}^8+11205
   % {\cal J}^6+27601 {\cal J}^4+15789 {\cal J}^2+3305}{1024 {\cal J}^7
   % \left({\cal J}^2+1\right)^{11/2}} \\
   % &-& \sum_{m>0,m\neq n}\frac{P_4(n,m,\cal J)}{16 {\cal J}^7 \left({\cal J}^2+1\right)^3 (m^2-n^2)^6 \left(m^2+n^2
   % {\cal J}^2\right)^{7/2}} \nonumber
% \eeqa
% where
% \beqa
% P_3(n,m,{\cal J})=&+&
% m^{10} n^3 \left(4 {\cal J}^4+11 {\cal J}^2+6\right)+2 m^8 n^5 \left(3
   % {\cal J}^6+5 {\cal J}^4-6 {\cal J}^2-6\right)\\
% \nn&+&2 m^6 n^7 \left({\cal J}^8-4 {\cal J}^6-11 {\cal J}^4+6 {\cal J}^2+9\right)+2 m^4
   % n^9 \left(-2 {\cal J}^8+9 {\cal J}^6+29 {\cal J}^4+14
   % {\cal J}^2-2\right)\\&+&m^2 n^{11} {\cal J}^2 \left(10 {\cal J}^6+16
% \nn   {\cal J}^4-2 {\cal J}^2-7\right)\;,
% \\P_4(n,m,{\cal J})=
% &+&4 m^{16} n^3 \left(8 {\cal J}^8+42 {\cal J}^6+85 {\cal J}^4+68
   % {\cal J}^2+20\right)
% \\&+&m^{14} n^5 \left(80 {\cal J}^{10}+302 {\cal J}^8+199
   % {\cal J}^6-703 {\cal J}^4-936 {\cal J}^2-340\right)
% \nn\\&+&m^{12} n^7 \left(64
   % {\cal J}^{12}-12 {\cal J}^{10}-893 {\cal J}^8-1765 {\cal J}^6+151
   % {\cal J}^4+1587 {\cal J}^2+740\right)
% \nn\\&+&m^{10} n^9 \left(16 {\cal J}^{14}-222
   % {\cal J}^{12}-587 {\cal J}^{10}+1209 {\cal J}^8+5444 {\cal J}^6+4374
   % {\cal J}^4+388 {\cal J}^2-520\right)
% \nn\\&+&2 m^8 n^{11} \left(-38 {\cal J}^{14}+200
   % {\cal J}^{12}+1446 {\cal J}^{10}+2505 {\cal J}^8+769 {\cal J}^6-511
   % {\cal J}^4+17 {\cal J}^2+210\right)
% \nn\\&+&m^6 n^{13} \left(200 {\cal J}^{14}+572
   % {\cal J}^{12}+206 {\cal J}^{10}-176 {\cal J}^8+2199 {\cal J}^6+3085
   % {\cal J}^4+1068 {\cal J}^2-60\right)
% \nn\\&+&m^4 n^{15} {\cal J}^2 \left(-76
   % {\cal J}^{12}+464 {\cal J}^{10}+2920 {\cal J}^8+5315 {\cal J}^6+3667
   % {\cal J}^4+643 {\cal J}^2-173\right)
% \nn\\&+&m^2 n^{17} {\cal J}^4 \left(256
   % {\cal J}^{10}+962 {\cal J}^8+1221 {\cal J}^6+401 {\cal J}^4-250
   % {\cal J}^2-148\right)\;.\nn
% \eeqa
% The expansion \eq{delta_oneloop_sj} can also be written in higher orders of $\mathcal{S}$ and $\mathcal{J}$, for $n=1$ we get
% {\small
% \beqa
	% &&\Delta_{1-loop} = \( \frac{-1}{2\, \mathcal{J}} + \frac{\mathcal{J}}{2} \) \, \mathcal{S} + \( \frac{1}{2 \, \mathcal{J}^3} - \[ \frac{3 \, \zeta_3}{2} + \frac{1}{16} \] \frac{1}{ \mathcal{J}} + \[\frac{3 \, \zeta_3}{2} + \frac{15 \, \zeta_5}{8} - \frac{21}{32} \] \, \mathcal{J}  \) \, \mathcal{S}^2  \\
	% &+& \( \frac{-3}{4 \, \mathcal{J}^5} + \[ \frac{3 \, \zeta_3}{2} + \frac{3}{16} \] \frac{1}{ \mathcal{J}^3} + \[ \frac{9 \, \zeta_3}{8} - \frac{1}{32} \] \frac{1}{\mathcal{J}} + \[ \frac{5}{4} - \frac{17 \, \zeta_3}{4} - \frac{65 \, \zeta_5}{16} - \frac{35 \, \zeta_7}{16} \] \, \mathcal{J}  \) \, \mathcal{S}^3\nonumber
% \\  &+&\(
% \frac{5}{4 {\cal J}^7}-
% \[\frac{7}{32}+\frac{9 \zeta_3}{4}\]\frac{1}{{\cal J}^5}
% -\[{\frac{3 \zeta_3}{4}-\frac{15 \zeta_5}{16}+\frac{5}{32}}\]\frac{1}{{\cal J}^3}-
% \[{\frac{145 \zeta_3}{64}+\frac{45 \zeta_5}{32}+\frac{175 \zeta_7}{128}+\frac{27}{1024}}\]\frac{1}{{\cal J}}
   % \){\cal S}^4 + \mathcal{O}(\mathcal{S}^5)\;,\nn
% \eeqa}
% for $n=2$,
% {\small
% \beqa
	% &&\Delta_{1-loop} = \( \frac{-1}{2\, \mathcal{J}} + \frac{\mathcal{J}}{2} \) \, \mathcal{S} + \( \frac{1}{2 \, \mathcal{J}^3} - \[ 12 \, \zeta_3 + \frac{17}{16} \] \frac{1}{ \mathcal{J}} + \[12 \, \zeta_3 + 60 \, \zeta_5 + \frac{27}{32} \] \, \mathcal{J}  \) \, \mathcal{S}^2   \\
	% &+& \( \frac{21}{4 \, \mathcal{J}^5} + \[ 12 \, \zeta_3 + \frac{19}{16} \] \frac{1}{ \mathcal{J}^3} + \[ 9 \, \zeta_3 + \frac{47}{32} \] \frac{1}{\mathcal{J}} - \[ \frac{19}{4} + 34 \, \zeta_3 + 130 \, \zeta_5 + 280 \, \zeta_7 \] \, \mathcal{J}  \) \, \mathcal{S}^3 \nonumber\\
  % &+&\(
% -\frac{175}{4 {\cal J}^7}-
% \[\frac{727}{32}+18 \zeta_3\]\frac{1}{{\cal J}^5}
% -\[{{6 \zeta_3}{}-30 \zeta_5-\frac{155}{32}}\]\frac{1}{{\cal J}^3}-
% \[{\frac{145 \zeta_3}{8}+{45 \zeta_5}+{175 \zeta_7}+\frac{7419}{1024}}\]\frac{1}{{\cal J}}
   % \){\cal S}^4 + \mathcal{O}(\mathcal{S}^5)\;,\nn
% \eeqa}
% and finally for $n=3$,
% \beqa
	% &&\Delta_{1-loop} = \( \frac{-1}{2\, \mathcal{J}} + \frac{\mathcal{J}}{2} \) \, \mathcal{S} + \( \frac{-5}{8 \, \mathcal{J}^3} - \[ \frac{81 \, \zeta_3}{2} + \frac{7}{4} \] \frac{1}{ \mathcal{J}} + \[\frac{81 \, \zeta_3}{2} + \frac{3645 \, \zeta_5}{8} - \frac{147}{64} \] \, \mathcal{J}  \) \, \mathcal{S}^2  \\
	% &+& \( \frac{1245}{32 \, \mathcal{J}^5} + \[ \frac{81 \, \zeta_3}{2} + \frac{39}{16} \] \frac{1}{ \mathcal{J}^3} + \[ \frac{243 \, \zeta_3}{8} + \frac{89}{32} \] \frac{1}{\mathcal{J}} - \[ \frac{89}{8} + \frac{459 \, \zeta_3}{4} + \frac{15795 \, \zeta_5}{16} + \frac{76545 \, \zeta_7}{16} \] \, \mathcal{J}  \) \, \mathcal{S}^3 \nonumber \\
	% &-& \( \frac{258785}{512 \, \mathcal{J} ^7} + \[ \frac{243 \, \zeta_3}{4} + \frac{251423}{1024} \] \frac{1}{\mathcal{J}^5} - \[ \frac{3645 \, \zeta_5}{16} - \frac{81 \, \zeta_3}{4} + \frac{256229}{4096} \] \frac{1}{\mathcal{J} ^3} \right.  \nonumber \\
	% &+& \left. \frac{27 \, (907200 \, \zeta_7 + 103680 \, \zeta_5 + 18560 \, \zeta_3 + 13457)}{8192 \, \mathcal{J} } \) \, \mathcal{S}^4 + \mathcal{O}(\mathcal{S}^5). \nonumber
% \eeqa
% These one-loop expressions can now be compared to \eq{As}, which is determined based on the exact slope conjecture of \cite{Basso:2011rs}. For $n=1$ the conjectured expression matches the exact result expanded up to 3 orders in $\mathcal{S}$, which hints that the conjecture is true. However for $n=2$ and $n=3$ we found the following inconsistencies:
% \begin{itemize}
	% \item when $n=2$, the coefficient of $\mathcal{S}^3/\mathcal{J}^5$ is predicted to be $-3/4$, but is actually $21/4$,
	% \item when $n=2$, the coefficient of $\mathcal{S}^4/\mathcal{J}^7$ is predicted to be $\frac{5}{4}$, but is actually $-\frac{175}{4}$,
	% \item when $n=2$, the coefficient of $\mathcal{S}^4/\mathcal{J}^5$ is predicted to be $-\frac{55 + 576 \, \zeta_3}{32}$, but is actually $-\frac{727 + 576 \, \zeta_3}{32}$,
	% \item when $n=3$, the coefficient of $\mathcal{S}^2/\mathcal{J}^3$ is predicted to be $1/2$, but is actually $-5/8$,
	% \item when $n=3$, the coefficient of $\mathcal{S}^3/\mathcal{J}^5$ is predicted to be $-3/4$, but is actually $1245/32$,
	% \item when $n=3$, the coefficient of $\mathcal{S}^3/\mathcal{J}^3$ is predicted to be $\frac{3(5 + 108 \, \zeta_3)}{8}$, but is actually $\frac{3(13 + 216 \, \zeta_3)}{16}$,
	% \item when $n=3$, the coefficient of $\mathcal{S}^4/\mathcal{J}^7$ is predicted to be $\frac{5}{4}$, but is actually $-\frac{258785}{512}$,
	% \item when $n=3$, the coefficient of $\mathcal{S}^4/\mathcal{J}^5$ is predicted to be $-\frac{11 + 243 \, \zeta_3}{4}$, but is actually $-\frac{251423 + 62208	\, \zeta_3}{1024}$.
	
% \end{itemize}
% As stated before, this signals that the conjecture has to be modified for cases when $n>1$.
