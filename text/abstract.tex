% main.tex

\section*{Abstract}

\vspace{30pt}

In this thesis we discuss supersymmetric gauge theories, focusing on exact results achieved using methods of integrability. 
Historically quantum field theories have always been approached using perturbation theory, which implicitly assumes weak coupling.
However this assumption is not always true, for example the theory describing strong interactions, quantum chromodynamics, is strongly coupled at low energies thus rendering problems like calculating masses of nucleons beyond the reach of perturbation theory.
Supersymmetric theories on the other hand exhibit additional symmetries and structures that when properly exploited can yield new techniques for solving otherwise intractable problems.
The $\N=4$ supersymmetric Yang-Mills theory is a prime example of such a theory, it is known to be integrable in the planar limit, which opens up a wealth of techniques one can employ in order to find results in this limit valid at any value of the coupling. 

For the larger portion of the text we study the $\N=4$ super Yang-Mills theory in the planar limit, a recurring concept being the Konishi anomalous dimension, which is roughly the analogue for the mass of the proton in quantum chromodynamics.
As expected we begin with perturbation theory where the integrability of the theory first manifests itself.
Here we are able to find the first exact result, the so-called slope function, which is the linear small spin expansion coefficient of the generalized Konishi anomalous dimension.
We then move on to exact results mainly achieved using the novel $\pmu$-system approach, which is a method that allows one to find the scaling dimensions of arbitrary operators at any values of the coupling.
As an example, we are able to find the second coefficient in the small spin expansion after the slope, which we call the curvature function.
This exact result allows us to extract non-trivial information about the Konishi operator.

Methods of integrability are also applicable to other supersymmetric gauge theories, one notable example being ABJM.
It is also known to be integrable and in fact shares many similarities with $\N=4$ super Yang-Mills, which we briefly review.
