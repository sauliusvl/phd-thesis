% main.tex

\section*{Abstract}

\vspace{30pt}

In this thesis we discuss supersymmetric gauge theories, focusing on exact results achieved using methods of integrability. 
For the larger part of the work we focus solely on the best known example of an integrable supersymmetric gauge theory, namely the $\N=4$ super Yang-Mills theory.
After motivating the problem of solving it exactly and briefly reviewing the historical developments in the subject we begin rigorous analysis by firstly defining the theory. 
We discuss it from the gauge theory point of view and give the alternative string theoretic formulation via the AdS/CFT correspondence.

We devote a large portion of the thesis for perturbative integrability, which addresses the spectral problem in perturbation theory both in weak and strong coupling. 
As we develop integrability technology we keep on applying new concepts to the simplest non-trivial observable in the theory, the Konishi operator.
We find its anomalous dimension using perturbative integrability methods at both ends of the range of the coupling constant.
At this point we find first hints of exact results by introducing the slope function and exploiting it to extend perturbative results for the Konishi operator.

Finally we address the most recent developments in solving the spectral problem exactly.
To that end we introduce the $\pmu$-system after shortly discussing its motivation using the now arguably obsolete TBA and Y-system approaches. 
We then demonstrate the applicability of this construction by rederiving the slope function and deriving the so called curvature function.
We show how these exact analytical results can be used to extract further information about the Konishi anomalous dimension. 

We devote a separate section to the ABJM theory, which is another example of an integrable supersymmetric gauge theory closely related to $\N=4$ supersymmetric Yang-Mills. 
As most of the techniques in ABJM are analogous to the ones describes while covering $\N=4$ SYM, we only provide a very brief review of results with references to other work containing the details.
